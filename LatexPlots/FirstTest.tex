%\documentclass[12pt, a4paper , draft]{report}
\documentclass[12pt, a4paper]{report}
\usepackage[utf8]{inputenc}
\usepackage{a4}
%\usepackage[none]{hyphenat} %hyphenation

\usepackage{chngcntr}
\counterwithout{footnote}{chapter}

\usepackage[bottom,flushmargin]{footmisc}
\usepackage{setspace}
\usepackage[pdfborder={0 0 0}]{hyperref}
\usepackage{fancyhdr}
\sloppy
%TODO need to decide if i want indentation or not
\usepackage{parskip} %no indentation after paragraphs    
%\usepackage{umlaute}
%\usepackage{afterpage} %for using \afterpage{\clearpage} (don't push images to the end of a chapter)
%\usepackage{makeidx}
%\usepackage[numbers]{natbib}
\usepackage{graphicx}
%\usepackage{picins} %provides precise control over the placement of inline graphics

\usepackage{titlesec}
%\usepackage{dsfont} %math symbols
\usepackage{tabularx}
\usepackage{wrapfig}
\usepackage{gensymb}
\usepackage{caption} %used to do \\ during caption
\usepackage{enumitem}% http://ctan.org/pkg/enumitem

% Florian Schulze, 06.06.2012
% v1.0, latest edit: 06.06.2012

%\usepackage{enumitem} %resume counting from previous enumerate block
%\usepackage{amsmath,amssymb}
%\usepackage[format=default,font=footnotesize,labelfont=bf]{caption}
%\usepackage{listings} %for listing source code
%\usepackage{color}
%\usepackage{algpseudocode} %for listing pseudocode
%\usepackage{algorithm} %wrap algpseudocode and enrich with label etc.
%\usepackage{float} % for [H] after floats

\usepackage{listings}
\usepackage{color}
\usepackage{pgfplots}
\usepgfplotslibrary{colormaps}

%for external pgfplots: https://tex.stackexchange.com/questions/7953/how-to-expand-texs-main-memory-size-pgfplots-memory-overload
%\usepgfplotslibrary{external} 
%\tikzexternalize

\definecolor{dkgreen}{rgb}{0,0.6,0}
\definecolor{gray}{rgb}{0.5,0.5,0.5}
\definecolor{mauve}{rgb}{0.58,0,0.82}

\lstset{ %
	language=[Sharp]C,                % choose the language of the code
	basicstyle=\footnotesize,       % the size of the fonts that are used for the code
	numbers=left,                   % where to put the line-numbers
	numberstyle=\footnotesize,      % the size of the fonts that are used for the line-numbers
	stepnumber=1,                   % the step between two line-numbers. If it is 1 each line will be numbered
	numbersep=7pt,                  % how far the line-numbers are from the code
	backgroundcolor=\color{white},  % choose the background color. You must add \usepackage{color}
	showspaces=false,               % show spaces adding particular underscores
	showstringspaces=false,         % underline spaces within strings
	showtabs=false,                 % show tabs within strings adding particular underscores
	frame=single,           % adds a frame around the code
	tabsize=2,          % sets default tabsize to 2 spaces
	captionpos=b,           % sets the caption-position to bottom
	breaklines=true,        % sets automatic line breaking
	breakatwhitespace=false,    % sets if automatic breaks should only happen at whitespace
	escapeinside={\%*}{*)},          % if you want to add a comment within your code
	columns=fullflexible,
	xleftmargin=0.5cm
}

\newcommand{\branchf}{Branching Factor}
\newcommand{\nodes}{Node Size}
\newcommand{\leafs}{Leaf Size}
\newcommand{\leafi}{Leaf Intersection}
\newcommand{\nodei}{Node Intersection}
\newcommand{\sleafi}{Shadow Leaf Intersection}
\newcommand{\snodei}{Shadow Node Intersection}
\newcommand{\cost}{Cost Function}
\newcommand{\scost}{Shadow Cost Function}
\newcommand{\sah}{Sah}
\newcommand{\epo}{Epo}
\newcommand{\waste}{Waste Factor}

%plots a value for given scene: titleName, rowName , fileName
\newcommand{\plot} [3]
{
	\begin{tikzpicture}
	\begin{axis}
	[
	xlabel = \leafs,
	ylabel = \branchf,
	colorbar style={title= #1}
	]
	\addplot[matrix plot*,point meta=\thisrow{#2}]
	table[x = leafSize, y = branchFactor, col sep=comma]{#3};
	\end{axis}
	\end{tikzpicture}
}

%plots a value for given scene: titleName, rowName , fileName   -> is for right plots without the label info on the left side
\newcommand{\plotr} [3]
{
	\begin{tikzpicture}
	\begin{axis}
	[
	xlabel = \leafs,
	colorbar style={title= #1}
	]
	\addplot[matrix plot*,point meta=\thisrow{#2}]
	table[x = leafSize, y = branchFactor, col sep=comma]{#3};
	\end{axis}
	\end{tikzpicture}
}

%possible color sceme. More suitable for color bars, for lines i use exotic since the colors are less bright
% Accessible colors from https://www.idpwd.com.au/resources/style-guide/
%from sebastian neubauer 
\definecolor{color0}{RGB}{247,127,0}
\definecolor{color1}{RGB}{81,181,224}
\definecolor{color2}{RGB}{206,224,7}
\definecolor{color3}{RGB}{15,43,91}

\definecolor{textcolor0}{RGB}{0,0,0}
\definecolor{textcolor1}{RGB}{0,0,0}
\definecolor{textcolor2}{RGB}{0,0,0}
\definecolor{textcolor3}{RGB}{255,255,255}

\pgfplotscreateplotcyclelist{mycolors}{
	{draw=color0},
	{draw=color1},
	{draw=color2},
	{draw=color3},
}

%plots a value for all scenenes titleName, rowName
\newcommand{\plotValue} [2]
{
	\begin{figure}[!htb]
		\begin{minipage}[t]{0.4\textwidth} 
			\plot{#1  shift happens}{#2}{Data/shiftHappensTableWithSpaceSorted.txt}
		\end{minipage}\hfil \hfil
		\begin{minipage}[t]{0.4\textwidth}
			%TODO move label to the right side?
			\plot{#1  sponza}{#2}{Data/sponzaTableWithSpaceSorted.txt}
		\end{minipage}
		
		\begin{minipage}[t]{0.4\textwidth} 
			\plot{#1  rungholt}{#2}{Data/rungholtTableWithSpaceSorted.txt}
		\end{minipage}\hfil \hfil
		\begin{minipage}[t]{0.4\textwidth}
			%TODO move label to the right side?
			\plot{#1  erato}{#2}{Data/eratoTableWithSpaceSorted.txt}
		\end{minipage}
	
		\begin{minipage}[t]{0.4\textwidth} 
			\plot{#1  average}{#2}{Data/AverageTableWithSpaceSorted.txt}
		\end{minipage}\hfil \hfil
		
		
		\caption{#1 of all scenes.}
	\end{figure}
	\newpage
}

%plots a selection of intresting values for given scene
\newcommand{\plotAll} [2]
{
\begin{figure}[!htb]
	\begin{minipage}[t]{0.4\textwidth} 
		\begin{tikzpicture}
		\begin{axis}
		[
		xlabel = \leafs,
		ylabel = \branchf,
		colorbar style={title=\leafi}
		]
		\addplot[matrix plot*,point meta=\thisrow{primaryLeafIntersections}]
		table[x = leafSize, y = branchFactor, col sep=comma]{#1};
		\end{axis}
		\end{tikzpicture}
	\end{minipage}\hfil \hfil
	\begin{minipage}[t]{0.4\textwidth}
		%TODO move label to the right side
		\begin{tikzpicture}
		\begin{axis}
		[
		xlabel = \leafs,
		%ylabel = branching factpr,
		colorbar style={title=\nodei}
		]
		\addplot[matrix plot*,point meta=\thisrow{primaryNodeIntersections}]
		table[x = leafSize, y = branchFactor, col sep=comma]{#1};
		\end{axis}
		\end{tikzpicture}
	\end{minipage}
	
	\begin{minipage}[t]{0.4\textwidth} 
		\begin{tikzpicture}
		\begin{axis}
		[
		xlabel = \leafs,
		ylabel = \branchf,
		colorbar style={title=\sleafi}
		]
		\addplot[matrix plot*,point meta=\thisrow{secondaryLeafIntersections}]
		table[x = leafSize, y = branchFactor, col sep=comma]{#1};
		\end{axis}
		\end{tikzpicture}
	\end{minipage}\hfil \hfil
	\begin{minipage}[t]{0.4\textwidth}
		%TODO move label to the right side
		\begin{tikzpicture}
		\begin{axis}
		[
		xlabel = \leafs,
		%ylabel = branching factpr,
		colorbar style={title=\snodei}
		]
		\addplot[matrix plot*,point meta=\thisrow{secondaryNodeIntersections}]
		table[x = leafSize, y = branchFactor, col sep=comma]{#1};
		\end{axis}
		\end{tikzpicture}
	\end{minipage}
	
	\begin{minipage}[t]{0.4\textwidth}
		\begin{tikzpicture}
		\begin{axis}
		[
		xlabel = \leafs,
		ylabel = \branchf,
		colorbar style={title=\cost}
		]
		\addplot[matrix plot*,point meta=\thisrow{PrimaryCost}]
		table[x = leafSize, y = branchFactor, col sep=comma]{#1};
		\end{axis}
		\end{tikzpicture}
	\end{minipage}\hfil\hfil
	\begin{minipage}[t]{0.4\textwidth}
		
		%TODO move label to the right side
		\begin{tikzpicture}
		\begin{axis}
		[
		xlabel = \leafs,
		%		ylabel = branching factpr,
		colorbar style={title=\scost}
		]
		\addplot[matrix plot*,point meta=\thisrow{SecondaryCost}]
		table[x = leafSize, y = branchFactor, col sep=comma]{#1};
		\end{axis}
		\end{tikzpicture}
	\end{minipage}

	\begin{minipage}[t]{0.4\textwidth}
		\begin{tikzpicture}
		\begin{axis}
		[
		xlabel = \leafs,
		ylabel = \branchf,
		colorbar style={title=\sah}
		]
		\addplot[matrix plot*,point meta=\thisrow{Sah}]
		table[x = leafSize, y = branchFactor, col sep=comma]{#1};
		\end{axis}
		\end{tikzpicture}
	\end{minipage}\hfil\hfil
	\begin{minipage}[t]{0.4\textwidth}
		%TODO move label to the right side
		\begin{tikzpicture}
		\begin{axis}
		[
		xlabel = \leafs,
		%		ylabel = branching factpr,
		colorbar style={title=\epo}
		]
		\addplot[matrix plot*,point meta=\thisrow{Epo}]
		table[x = leafSize, y = branchFactor, col sep=comma]{#1};
		\end{axis}
		\end{tikzpicture}
	\end{minipage}
	\caption{#2}
\end{figure}
\newpage
}

%fcommands to manage plots.
\newcommand{\file}{xxx}
\newcommand{\labelText}{xxx}

\titleformat{\paragraph}[hang]{\normalfont\bfseries}{\theparagraph}{.5em}{}

\pgfplotsset{
	colormap/viridis,
	every axis/.append style={
		scale only axis,
		width=0.8\textwidth,
		%		height = 5.9cm
		mark size=4pt,
		%point meta max = 10,
		colorbar,
		colormap={reverse viridis}{
			indices of colormap={
				\pgfplotscolormaplastindexof{viridis},...,0 of viridis}
		}
	}
}

%add discard of and discard of no option
\pgfplotsset{
	discard if/.style 2 args={
		x filter/.append code={
			\edef\tempa{\thisrow{#1}}
			\edef\tempb{#2}
			\ifx\tempa\tempb
			\def\pgfmathresult{inf}
			\fi
		}
	},
	discard if not/.style 2 args={
		x filter/.append code={
			\edef\tempa{\thisrow{#1}}
			\edef\tempb{#2}
			\ifx\tempa\tempb
			\else
			\def\pgfmathresult{inf}
			\fi
		}
	}
}

\begin{document}

\iffalse

Cost factor for nodes: 1, leafs: 1

This is used for Cost Function, Sah and Epo

Currently all scenes are rendered with Sorted and with efficient rendering (saving distance of aabb and not intersecting when the ray already found a closer triangle)


Nomenclature:

Nx = Node size of x (also called branching factor)

Lx = Leaf size of x
\fi

\iffalse
\section{RandomPlots}

\pgfplotsset{
	every axis/.append style={colorbar}
}

%overview over sah and epo.
\iftrue
Following 6 pages are the Sah and Epo for every scene.

A problem with the Surface are heuristic is that the leaf portion  of the heuristic is to large in comparison to the node portion. As an example, for N16L16 (rungholt) the leaf Sah is 177 and the node Sah is 15, but we have 2.3 leaf intersections with 10.4 node intersections. (it should be noted that the nodes and leafs both have a cost factor of 1)

The end point overlap heavily depends on the specific geometry of the scene. Scenes like rungholt and eratio have a very low epo in comparison to scenes like shift happens and sponza. The reason for this is that the rungholt and eratio consist out of evenly sized triangles, while the others use large triangles when there is no change in the surface geometry. Example for this are the walls and the floor of sponza.

Spikes in the leaf epo often correlate with higher leaf intersections caused by unfortunate splits of the scene, but its not guaranteed. The node part of the epo looks very similar

\newpage

\plotValue{Node Sah}{nodeSah}
\plotValue{Leaf Sah}{leafSah}
\plotValue{Sah}{Sah}
%node basically doesnt care about scenes
\plotValue{Node Epo}{nodeEpo}
%leaf Epo heavily depends on scene
\plotValue{Leaf Epo}{leafEpo}
\plotValue{Epo}{Epo}
\fi

\iftrue
Following 2 pages: Leaf volume and leaf surface area of the scenes. Values dont seem to correlate much with any results. The only thing you can see in the surface area that most scenes are better dividable by even numbers.
\newpage

\plotValue{Leaf Volume}{leafVolume}
\plotValue{Leaf surface area}{leafSurfaceArea}
\fi

\iftrue
\plotValue{waste factor}{primaryWasteFactor}
Average tree depth explains slight arc in erato plots.
\plotValue{Average leaf depth}{averageLeafDepth}
\fi
\fi

\iffalse

\section{First analysis and selection of interesting configurations}

General knowledge from previous plots not mentioned here: The number of leaf intersections is very low in comparison to the number of Node intersections. This is good because the Leaf intersections behave very randomly depending on scene and Leaf Size, and dont change with the Branching factor. Therefore we try to pick a good Branching factor to minimize the Node intersections and dont have to worry about the Leaf intersections much.



\newpage


%block that compares some values of the average tables:
\iftrue
\begin{figure}[!htb]
	\begin{minipage}[t]{0.4\textwidth} 
		\plot{Node memory loaded}{primaryIntersectionMultBranch}{Data/AverageTableWithSpaceSorted.txt}
	\end{minipage}\hfil \hfil
	\begin{minipage}[t]{0.4\textwidth}
		\plotr{Leaf memory loaded}{primaryIntersectionsMultLeaf}{Data/AverageTableWithSpaceSorted.txt}
	\end{minipage}
	
	\begin{minipage}[t]{0.4\textwidth} 
		\plot{primary Node intersections}{primaryNodeIntersections}{Data/AverageTableWithSpaceSorted.txt}
	\end{minipage}\hfil \hfil
	\begin{minipage}[t]{0.4\textwidth}
		\plotr{primary Leaf intersections}{primaryLeafIntersections}{Data/AverageTableWithSpaceSorted.txt}
	\end{minipage}

	\begin{minipage}[t]{0.4\textwidth} 
		\plot{secondary Node intersections}{secondaryNodeIntersections}{Data/AverageTableWithSpaceSorted.txt}
	\end{minipage}\hfil \hfil
	\begin{minipage}[t]{0.4\textwidth}
		\plotr{secondary Leaf intersections}{secondaryLeafIntersections}{Data/AverageTableWithSpaceSorted.txt}
	\end{minipage}
	
	\caption{Overview of important values on the normalized average of all scenes.}
\end{figure}
\newpage
\fi

\pgfplotsset{
	every axis/.append style={colorbar = false}
}


\begin{figure}[!htb]
	\begin{minipage}[t]{0.4\textwidth}
		\begin{tikzpicture}
		\begin{axis}
		[
		%view={90}{0} for x, view={0}{0} for y restriction
		view={0}{0},
		xlabel = \leafs,
		ylabel = \branchf,
		zlabel = loaded Node Memory,
		cycle list name=exotic,
		xtick={2,3,4,5,6,7,8,9,10,11,12,13,14,15,16},
		xticklabels={,,,$5$,,,,,$10$,,,,,$15$,},
		]
		\addplot3+[thick, restrict y to domain={2: 2}, mark=none]table[x = leafSize, y = branchFactor, z = primaryIntersectionMultBranch, col sep=comma]{Data/AverageTableSorted.txt};
		\addplot3+[thick, restrict y to domain={3: 3}, mark=none]table[x = leafSize, y = branchFactor, z = primaryIntersectionMultBranch, col sep=comma]{Data/AverageTableSorted.txt};
		\addplot3+[thick, restrict y to domain={4: 4}, mark=none]table[x = leafSize, y = branchFactor, z = primaryIntersectionMultBranch, col sep=comma]{Data/AverageTableSorted.txt};
		
		\legend{N = 2, N = 3, N = 4}
		\end{axis}
		\end{tikzpicture}
	\end{minipage}\hfil \hfil
	\begin{minipage}[t]{0.4\textwidth}
		%TODO move label to the right side
		\begin{tikzpicture}
		\begin{axis}
		[
		%view={90}{0} for x, view={0}{0} for y restriction
		view={0}{0},
		xlabel = \leafs,
		ylabel = \branchf,
		zlabel = \nodei,
		cycle list name=exotic,
		legend style={at={(0.99,1.1)}, anchor = north east},
		xtick={2,3,4,5,6,7,8,9,10,11,12,13,14,15,16},
		xticklabels={,,,$5$,,,,,$10$,,,,,$15$,},
		]
		\addplot3+[thick, restrict y to domain={2: 2}, mark=none]table[x = leafSize, y = branchFactor, z = primaryNodeIntersections, col sep=comma]{Data/AverageTableSorted.txt};
		\addplot3+[thick, restrict y to domain={3: 3}, mark=none]table[x = leafSize, y = branchFactor, z = primaryNodeIntersections, col sep=comma]{Data/AverageTableSorted.txt};
		\addplot3+[thick, restrict y to domain={4: 4}, mark=none]table[x = leafSize, y = branchFactor, z = primaryNodeIntersections, col sep=comma]{Data/AverageTableSorted.txt};
		
		\legend{N = 2, N = 3, N = 4}
		\end{axis}
		\end{tikzpicture}
	\end{minipage}
	
	\caption{Comparison of N2, N3 and N4 with the normalized average result. This shows that N=3 is significantly better than N2. It only loads minimally more Node Memory but has about 2/3 of the node Intersections of N2.}
\end{figure}

\begin{figure}[!htb]
	\begin{minipage}[t]{0.4\textwidth}
		\begin{tikzpicture}
		\begin{axis}
		[
		%view={90}{0} for x, view={0}{0} for y restriction
		view={90}{00},
		xlabel = \leafs,
		ylabel = \branchf,
		zlabel = loaded Node Memory,
		legend style={at={(0.01,0.99)}, anchor = north west},
		cycle list name=exotic,
		ytick={2,3,4,5,6,7,8,9,10,11,12,13,14,15,16},
		yticklabels={,,,$5$,,,,,$10$,,,,,$15$,},
		]
		\addplot3+[thick, restrict x to domain={1:1}, unbounded coords=discard, mark=none]table[x = leafSize, y = branchFactor, z = primaryIntersectionMultBranch, col sep=comma]{Data/AverageTableSorted.txt};
		\addplot3+[thick, restrict x to domain={2:2}, unbounded coords=discard, mark=none]table[x = leafSize, y = branchFactor, z = primaryIntersectionMultBranch, col sep=comma]{Data/AverageTableSorted.txt};
		\addplot3+[thick, restrict x to domain={3:3}, unbounded coords=discard, mark=none]table[x = leafSize, y = branchFactor, z = primaryIntersectionMultBranch, col sep=comma]{Data/AverageTableSorted.txt};
		\addplot3+[thick, restrict x to domain={4:4}, unbounded coords=discard, mark=none]table[x = leafSize, y = branchFactor, z = primaryIntersectionMultBranch, col sep=comma]{Data/AverageTableSorted.txt};
		
		\legend{L1,L2,L3,L4}
		\end{axis}
		\end{tikzpicture}
	\end{minipage}\hfil \hfil
	\begin{minipage}[t]{0.4\textwidth}
		%TODO move label to the right side
		\begin{tikzpicture}
		\begin{axis}
		[
		%view={90}{0} for x, view={0}{0} for y restriction
		view={90}{0},
		xlabel = \leafs,
		ylabel = \branchf,
		zlabel = \nodei,
		cycle list name=exotic,
		ytick={2,3,4,5,6,7,8,9,10,11,12,13,14,15,16},
		yticklabels={,,,$5$,,,,,$10$,,,,,$15$,},
		]
		\addplot3+[thick, restrict x to domain={1:1}, unbounded coords=discard, mark=none]table[x = leafSize, y = branchFactor, z = primaryNodeIntersections, col sep=comma]{Data/AverageTableSorted.txt};
		\addplot3+[thick, restrict x to domain={2:2}, unbounded coords=discard, mark=none]table[x = leafSize, y = branchFactor, z = primaryNodeIntersections, col sep=comma]{Data/AverageTableSorted.txt};
		\addplot3+[thick, restrict x to domain={3:3}, unbounded coords=discard, mark=none]table[x = leafSize, y = branchFactor, z = primaryNodeIntersections, col sep=comma]{Data/AverageTableSorted.txt};
		\addplot3+[thick, restrict x to domain={4:4}, unbounded coords=discard, mark=none]table[x = leafSize, y = branchFactor, z = primaryNodeIntersections, col sep=comma]{Data/AverageTableSorted.txt};
		
		\legend{L1,L2,L3,L4}
		\end{axis}
		\end{tikzpicture}
	\end{minipage}
	\caption{Comparison of L1, L2, L3, L4 with the normalized average result. No significant difference between the different leafsizes. The loaded Node memory is lower for higher leafsizes because the Bvh is smaller. Those two plots also visualize the effect of increasing the branching factor. It increases the memory usage linearly, but the improvement in Node intersections is asymptotic. }
\end{figure}

\begin{figure}[!htb]
	\begin{minipage}[t]{0.4\textwidth}
		\begin{tikzpicture}
		\begin{axis}
		[
		%view={90}{0} for x, view={0}{0} for y restriction
		view={0}{0},
		xlabel = \leafs,
		ylabel = \branchf,
		zlabel = loaded Node Memory,
		cycle list name=exotic,
		legend style={at={(0.99,1.3)}, anchor = north east},
		xtick={2,3,4,5,6,7,8,9,10,11,12,13,14,15,16},
		xticklabels={,,,$5$,,,,,$10$,,,,,$15$,},
		]
		\addplot3+[thick, restrict y to domain={4: 4}, mark=none]table[x = leafSize, y = branchFactor, z = primaryIntersectionMultBranch, col sep=comma]{Data/AverageTableSorted.txt};
		\addplot3+[thick, restrict y to domain={5: 5}, mark=none]table[x = leafSize, y = branchFactor, z = primaryIntersectionMultBranch, col sep=comma]{Data/AverageTableSorted.txt};
		\addplot3+[thick, restrict y to domain={6: 6}, mark=none]table[x = leafSize, y = branchFactor, z = primaryIntersectionMultBranch, col sep=comma]{Data/AverageTableSorted.txt};
		\addplot3+[thick, restrict y to domain={7: 7}, mark=none]table[x = leafSize, y = branchFactor, z = primaryIntersectionMultBranch, col sep=comma]{Data/AverageTableSorted.txt};
		\addplot3+[thick, restrict y to domain={8: 8}, mark=none]table[x = leafSize, y = branchFactor, z = primaryIntersectionMultBranch, col sep=comma]{Data/AverageTableSorted.txt};
		
		\legend{N = 4, N = 5, N = 6, N = 7, N = 8}
		\end{axis}
		\end{tikzpicture}
	\end{minipage}\hfil \hfil
	\begin{minipage}[t]{0.4\textwidth}
		%TODO move label to the right side
		\begin{tikzpicture}
		\begin{axis}
		[
		%view={90}{0} for x, view={0}{0} for y restriction
		view={0}{0},
		xlabel = \leafs,
		ylabel = \branchf,
		zlabel = \nodei,
		cycle list name=exotic,
		legend style={at={(0.99,1.3)}, anchor = north east},
		xtick={2,3,4,5,6,7,8,9,10,11,12,13,14,15,16},
		xticklabels={,,,$5$,,,,,$10$,,,,,$15$,},
		]
		\addplot3+[thick, restrict y to domain={4: 4}, mark=none]table[x = leafSize, y = branchFactor, z = primaryNodeIntersections, col sep=comma]{Data/AverageTableSorted.txt};
		\addplot3+[thick, restrict y to domain={5: 5}, mark=none]table[x = leafSize, y = branchFactor, z = primaryNodeIntersections, col sep=comma]{Data/AverageTableSorted.txt};
		\addplot3+[thick, restrict y to domain={6: 6}, mark=none]table[x = leafSize, y = branchFactor, z = primaryNodeIntersections, col sep=comma]{Data/AverageTableSorted.txt};
		\addplot3+[thick, restrict y to domain={7: 7}, mark=none]table[x = leafSize, y = branchFactor, z = primaryNodeIntersections, col sep=comma]{Data/AverageTableSorted.txt};
		\addplot3+[thick, restrict y to domain={8: 8}, mark=none]table[x = leafSize, y = branchFactor, z = primaryNodeIntersections, col sep=comma]{Data/AverageTableSorted.txt};
		
		\legend{N = 4, N = 5, N = 6, N = 7, N = 8}
		\end{axis}
		\end{tikzpicture}
	\end{minipage}
	
	\caption{Comparison of N4, N5, N6, N7, N8 with the normalized average result. Here we can observe the effect that the difference between an odd N and the next larger even N is smaller than of a even N to the next larger odd N. Therefore it should be better to choose odd numbers like 5 and 7 than even numbers like 3, 6, and 8.}
\end{figure}
\newpage
\begin{figure}[!htb]
	\begin{minipage}[t]{0.4\textwidth}
		\begin{tikzpicture}
		\begin{axis}
		[
		%view={90}{0} for x, view={0}{0} for y restriction
		view={0}{0},
		xlabel = \leafs,
		ylabel = \branchf,
		zlabel = loaded Leaf Memory,
		cycle list name=exotic,
		legend style={at={(0.99,1.3)}, anchor = north east},
		xtick={2,3,4,5,6,7,8,9,10,11,12,13,14,15,16},
		xticklabels={,,,$5$,,,,,$10$,,,,,$15$,},
		]
		\addplot3+[thick, restrict y to domain={1: 1}, mark=none]table[x = leafSize, y = branchFactor, z = primaryIntersectionsMultLeaf, col sep=comma]{Data/AverageTableSorted.txt};
		\addplot3+[thick, restrict y to domain={2: 2}, mark=none]table[x = leafSize, y = branchFactor, z = primaryIntersectionsMultLeaf, col sep=comma]{Data/AverageTableSorted.txt};
		\addplot3+[thick, restrict y to domain={3: 3}, mark=none]table[x = leafSize, y = branchFactor, z = primaryIntersectionsMultLeaf, col sep=comma]{Data/AverageTableSorted.txt};
		\addplot3+[thick, restrict y to domain={4: 4}, mark=none]table[x = leafSize, y = branchFactor, z = primaryIntersectionsMultLeaf, col sep=comma]{Data/AverageTableSorted.txt};
		\addplot3+[thick, restrict y to domain={5: 5}, mark=none]table[x = leafSize, y = branchFactor, z = primaryIntersectionsMultLeaf, col sep=comma]{Data/AverageTableSorted.txt};
		\addplot3+[thick, restrict y to domain={6: 6}, mark=none]table[x = leafSize, y = branchFactor, z = primaryIntersectionsMultLeaf, col sep=comma]{Data/AverageTableSorted.txt};
		\addplot3+[thick, restrict y to domain={7: 7}, mark=none]table[x = leafSize, y = branchFactor, z = primaryIntersectionsMultLeaf, col sep=comma]{Data/AverageTableSorted.txt};
		\addplot3+[thick, restrict y to domain={8: 8}, mark=none]table[x = leafSize, y = branchFactor, z = primaryIntersectionsMultLeaf, col sep=comma]{Data/AverageTableSorted.txt};
		
		\end{axis}
		\end{tikzpicture}
	\end{minipage}\hfil \hfil
	\begin{minipage}[t]{0.4\textwidth}
		%TODO move label to the right side
		\begin{tikzpicture}
		\begin{axis}
		[
		%view={90}{0} for x, view={0}{0} for y restriction
		view={0}{0},
		xlabel = \leafs,
		ylabel = \branchf,
		zlabel = \leafi,
		cycle list name=exotic,
		legend style={at={(0.99,1.3)}, anchor = north east},
		xtick={2,3,4,5,6,7,8,9,10,11,12,13,14,15,16},
		xticklabels={,,,$5$,,,,,$10$,,,,,$15$,},
		]
		\addplot3+[thick, restrict y to domain={1: 1}, mark=none]table[x = leafSize, y = branchFactor, z = primaryLeafIntersections, col sep=comma]{Data/AverageTableSorted.txt};
		\addplot3+[thick, restrict y to domain={2: 2}, mark=none]table[x = leafSize, y = branchFactor, z = primaryLeafIntersections, col sep=comma]{Data/AverageTableSorted.txt};
		\addplot3+[thick, restrict y to domain={3: 3}, mark=none]table[x = leafSize, y = branchFactor, z = primaryLeafIntersections, col sep=comma]{Data/AverageTableSorted.txt};
		\addplot3+[thick, restrict y to domain={4: 4}, mark=none]table[x = leafSize, y = branchFactor, z = primaryLeafIntersections, col sep=comma]{Data/AverageTableSorted.txt};
		\addplot3+[thick, restrict y to domain={5: 5}, mark=none]table[x = leafSize, y = branchFactor, z = primaryLeafIntersections, col sep=comma]{Data/AverageTableSorted.txt};
		\addplot3+[thick, restrict y to domain={6: 6}, mark=none]table[x = leafSize, y = branchFactor, z = primaryLeafIntersections, col sep=comma]{Data/AverageTableSorted.txt};
		\addplot3+[thick, restrict y to domain={7: 7}, mark=none]table[x = leafSize, y = branchFactor, z = primaryLeafIntersections, col sep=comma]{Data/AverageTableSorted.txt};
		\addplot3+[thick, restrict y to domain={8: 8}, mark=none]table[x = leafSize, y = branchFactor, z = primaryLeafIntersections, col sep=comma]{Data/AverageTableSorted.txt};
		
		\end{axis}
		\end{tikzpicture}
	\end{minipage}
	
	\caption{Comparison of N1 to N8 with the normalized average result. The different branching factor has no effect on the leaf Intersections. The Leaf memory usage in respect to the Leafsize behaves very similar to the Node Memory usage in respect to the Branching Factor. But the \leafi  are not very predictable. The odd numbers are better on most scenes (especially noticeable for rungholt and sponza), but the overall difference is minimal, so it seems the best to adjust the Leafsize to the hardware.}
\end{figure}

\newpage

Current selection of N and L:

N2L1 as the most generic version, and N4L4 for QBvh

N3L1, N3L2 and N3L4 to compare it to the generic versions since our data suggests that N3 is very beneficial

N5LX and N7LX as described in Figure 4. X stands for arbitrary Leaf size depending on hardware preferences. If i had to choose i would take either L2, L4 or L6 because it seems to be good to have even Leaf Sizes for all our scenes.
\fi

\iffalse

\section{First Ispc performance test}

This are the results for the sponza scene. 

Time measurements are done on the new raytracer that uses Ispc. It saves the distance of aabb intersections and stops early if ray already hit something closer. It currently only supports bvh with multiple split axis for the children.

"Time triangle intersections" is the time all triangle intersections took. "Time rest" means the total time all rays took to render the image minus the time the triangle intersections took. Both times are in seconds. To get more consistent data multi threading was not used. Both triangle intersection and aabb intersections are done with Ispc. Each memory block is padded to 32 bit (both triangles and aabb). Im sure the triangle intersection code can still be optimized.

The program was executed with windows priority "high" but results are not really consistent.

First set of plots is with avx (8 lanes) and the second set is with sse(4 lanes)

\iftrue
\begin{figure}[!htb]
	\begin{minipage}[t]{0.4\textwidth} 
		\plot{Time triangle intersections}{triangleIntersectionSum}{Data/sponzaPad8TableWithSpace.txt}
	\end{minipage}\hfil \hfil
	\begin{minipage}[t]{0.4\textwidth}
		\plotr{Time rest}{rayTimeWithoutTri}{Data/sponzaPad8TableWithSpace.txt}
	\end{minipage}
	
	\caption{Overview over the sponza ispc performance tests. (avx2)}
\end{figure}
\newpage
\fi

\pgfplotsset{
	every axis/.append style={colorbar = false}
}

\begin{figure}[!htb]
	\begin{minipage}[t]{0.4\textwidth}
		\begin{tikzpicture}
		\begin{axis}
		[
		%view={90}{0} for x, view={0}{0} for y restriction
		view={0}{0},
		xlabel = \leafs,
		ylabel = \branchf,
		zlabel = Time triangle intersections,
		cycle list name=exotic,
		legend style={at={(0.99,1.5)}, anchor = north east},
		xtick={2,3,4,5,6,7,8,9,10,11,12,13,14,15,16},
		xticklabels={,,,$5$,,,,,$10$,,,,,$15$,},
		]
		\addplot3+[thick, restrict y to domain={2: 2}, mark=none]table[x = leafSize, y = branchFactor, z = triangleIntersectionSum, col sep=comma]{Data/sponzaPad8Table.txt};
		\addplot3+[thick, restrict y to domain={4: 4}, mark=none]table[x = leafSize, y = branchFactor, z = triangleIntersectionSum, col sep=comma]{Data/sponzaPad8Table.txt};
		\addplot3+[thick, restrict y to domain={8: 8}, mark=none]table[x = leafSize, y = branchFactor, z = triangleIntersectionSum, col sep=comma]{Data/sponzaPad8Table.txt};
		\addplot3+[thick, restrict y to domain={16: 16}, mark=none]table[x = leafSize, y = branchFactor, z = triangleIntersectionSum, col sep=comma]{Data/sponzaPad8Table.txt};
		
		\legend{N2, N4, N8, N16}
		\end{axis}
		\end{tikzpicture}
	\end{minipage}\hfil \hfil
	\begin{minipage}[t]{0.4\textwidth}
		%TODO move label to the right side
		\begin{tikzpicture}
		\begin{axis}
		[
		%view={90}{0} for x, view={0}{0} for y restriction
		view={0}{0},
		xlabel = \leafs,
		ylabel = \branchf,
		zlabel = Time rest,
		cycle list name=exotic,
		legend style={at={(0.99,1.5)}, anchor = north east},
		xtick={2,3,4,5,6,7,8,9,10,11,12,13,14,15,16},
		xticklabels={,,,$5$,,,,,$10$,,,,,$15$,},
		]
		\addplot3+[thick, restrict y to domain={2: 2}, mark=none]table[x = leafSize, y = branchFactor, z = rayTimeWithoutTri, col sep=comma]{Data/sponzaPad8Table.txt};
		\addplot3+[thick, restrict y to domain={4: 4}, mark=none]table[x = leafSize, y = branchFactor, z = rayTimeWithoutTri, col sep=comma]{Data/sponzaPad8Table.txt};
		\addplot3+[thick, restrict y to domain={8: 8}, mark=none]table[x = leafSize, y = branchFactor, z = rayTimeWithoutTri, col sep=comma]{Data/sponzaPad8Table.txt};
		\addplot3+[thick, restrict y to domain={16: 16}, mark=none]table[x = leafSize, y = branchFactor, z = rayTimeWithoutTri, col sep=comma]{Data/sponzaPad8Table.txt};
		
		\legend{N2, N4, N8, N16}
		\end{axis}
		\end{tikzpicture}
	\end{minipage}
	
	\caption{Comparison of N2, N4, N8, N16 for sponza performance tests. (avx2)}
\end{figure}

\begin{figure}[!htb]
	\begin{minipage}[t]{0.4\textwidth}
		\begin{tikzpicture}
		\begin{axis}
		[
		%view={90}{0} for x, view={0}{0} for y restriction
		view={90}{00},
		xlabel = \leafs,
		ylabel = \branchf,
		zlabel = time triangle intersections,
		legend style={at={(0.99,1.5)}, anchor = north east},
		cycle list name=exotic,
		ytick={2,3,4,5,6,7,8,9,10,11,12,13,14,15,16},
		yticklabels={,,,$5$,,,,,$10$,,,,,$15$,},
		]
		\addplot3+[thick, restrict x to domain={4:4}, unbounded coords=discard, mark=none]table[x = leafSize, y = branchFactor, z = triangleIntersectionSum, col sep=comma]{Data/sponzaPad8Table.txt};
		\addplot3+[thick, restrict x to domain={8:8}, unbounded coords=discard, mark=none]table[x = leafSize, y = branchFactor, z = triangleIntersectionSum, col sep=comma]{Data/sponzaPad8Table.txt};
		\addplot3+[thick, restrict x to domain={12:12}, unbounded coords=discard, mark=none]table[x = leafSize, y = branchFactor, z = triangleIntersectionSum, col sep=comma]{Data/sponzaPad8Table.txt};
		\addplot3+[thick, restrict x to domain={16:16}, unbounded coords=discard, mark=none]table[x = leafSize, y = branchFactor, z = triangleIntersectionSum, col sep=comma]{Data/sponzaPad8Table.txt};
		
		\legend{L4,L8,L12,L16}
		\end{axis}
		\end{tikzpicture}
	\end{minipage}\hfil \hfil
	\begin{minipage}[t]{0.4\textwidth}
		%TODO move label to the right side
		\begin{tikzpicture}
		\begin{axis}
		[
		%view={90}{0} for x, view={0}{0} for y restriction
		view={90}{0},
		xlabel = \leafs,
		ylabel = \branchf,
		zlabel = time rest,
		cycle list name=exotic,
		ytick={2,3,4,5,6,7,8,9,10,11,12,13,14,15,16},
		yticklabels={,,,$5$,,,,,$10$,,,,,$15$,},
		]
		\addplot3+[thick, restrict x to domain={4:4}, unbounded coords=discard, mark=none]table[x = leafSize, y = branchFactor, z = rayTimeWithoutTri, col sep=comma]{Data/sponzaPad8Table.txt};
		\addplot3+[thick, restrict x to domain={8:8}, unbounded coords=discard, mark=none]table[x = leafSize, y = branchFactor, z = rayTimeWithoutTri, col sep=comma]{Data/sponzaPad8Table.txt};
		\addplot3+[thick, restrict x to domain={12:12}, unbounded coords=discard, mark=none]table[x = leafSize, y = branchFactor, z = rayTimeWithoutTri, col sep=comma]{Data/sponzaPad8Table.txt};
		\addplot3+[thick, restrict x to domain={16:16}, unbounded coords=discard, mark=none]table[x = leafSize, y = branchFactor, z = rayTimeWithoutTri, col sep=comma]{Data/sponzaPad8Table.txt};
		
		\legend{L4,L8,L12,L16}
		\end{axis}
		\end{tikzpicture}
	\end{minipage}
	\caption{Comparison of L4, L8, L12, L16 for sponza performance tests. (avx2) (Im sure the spikes on L4 are not consistent there. The run before padding had no anomalies like that?)}
\end{figure}

\pgfplotsset{
	every axis/.append style={colorbar = true}
}
\iftrue
\begin{figure}[!htb]
	\begin{minipage}[t]{0.4\textwidth} 
		\plot{Time triangle intersections}{triangleIntersectionSum}{Data/sponzaSeeTableWithSpace.txt}
	\end{minipage}\hfil \hfil
	\begin{minipage}[t]{0.4\textwidth}
		\plotr{Time rest}{rayTimeWithoutTri}{Data/sponzaSeeTableWithSpace.txt}
	\end{minipage}
	
	\caption{Overview over the sponza ispc performance tests. (see4)}
\end{figure}
\newpage
\fi

\pgfplotsset{
	every axis/.append style={colorbar = false}
}

\begin{figure}[!htb]
	\begin{minipage}[t]{0.4\textwidth}
		\begin{tikzpicture}
		\begin{axis}
		[
		%view={90}{0} for x, view={0}{0} for y restriction
		view={0}{0},
		xlabel = \leafs,
		ylabel = \branchf,
		zlabel = Time triangle intersections,
		cycle list name=exotic,
		legend style={at={(0.99,1.5)}, anchor = north east},
		xtick={2,3,4,5,6,7,8,9,10,11,12,13,14,15,16},
		xticklabels={,,,$5$,,,,,$10$,,,,,$15$,},
		]
		\addplot3+[thick, restrict y to domain={2: 2}, mark=none]table[x = leafSize, y = branchFactor, z = triangleIntersectionSum, col sep=comma]{Data/sponzaSeeTable.txt};
		\addplot3+[thick, restrict y to domain={4: 4}, mark=none]table[x = leafSize, y = branchFactor, z = triangleIntersectionSum, col sep=comma]{Data/sponzaSeeTable.txt};
		\addplot3+[thick, restrict y to domain={8: 8}, mark=none]table[x = leafSize, y = branchFactor, z = triangleIntersectionSum, col sep=comma]{Data/sponzaSeeTable.txt};
		\addplot3+[thick, restrict y to domain={16: 16}, mark=none]table[x = leafSize, y = branchFactor, z = triangleIntersectionSum, col sep=comma]{Data/sponzaSeeTable.txt};
		
		\legend{N2, N4, N8, N16}
		\end{axis}
		\end{tikzpicture}
	\end{minipage}\hfil \hfil
	\begin{minipage}[t]{0.4\textwidth}
		%TODO move label to the right side
		\begin{tikzpicture}
		\begin{axis}
		[
		%view={90}{0} for x, view={0}{0} for y restriction
		view={0}{0},
		xlabel = \leafs,
		ylabel = \branchf,
		zlabel = Time rest,
		cycle list name=exotic,
		legend style={at={(0.99,1.5)}, anchor = north east},
		xtick={2,3,4,5,6,7,8,9,10,11,12,13,14,15,16},
		xticklabels={,,,$5$,,,,,$10$,,,,,$15$,},
		]
		\addplot3+[thick, restrict y to domain={2: 2}, mark=none]table[x = leafSize, y = branchFactor, z = rayTimeWithoutTri, col sep=comma]{Data/sponzaSeeTable.txt};
		\addplot3+[thick, restrict y to domain={4: 4}, mark=none]table[x = leafSize, y = branchFactor, z = rayTimeWithoutTri, col sep=comma]{Data/sponzaSeeTable.txt};
		\addplot3+[thick, restrict y to domain={8: 8}, mark=none]table[x = leafSize, y = branchFactor, z = rayTimeWithoutTri, col sep=comma]{Data/sponzaSeeTable.txt};
		\addplot3+[thick, restrict y to domain={16: 16}, mark=none]table[x = leafSize, y = branchFactor, z = rayTimeWithoutTri, col sep=comma]{Data/sponzaSeeTable.txt};
		
		\legend{N2, N4, N8, N16}
		\end{axis}
		\end{tikzpicture}
	\end{minipage}
	
	\caption{Comparison of N2, N4, N8, N16 for sponza performance tests. (see4)}
\end{figure}

\begin{figure}[!htb]
	\begin{minipage}[t]{0.4\textwidth}
		\begin{tikzpicture}
		\begin{axis}
		[
		%view={90}{0} for x, view={0}{0} for y restriction
		view={90}{00},
		xlabel = \leafs,
		ylabel = \branchf,
		zlabel = time triangle intersections,
		legend style={at={(0.99,1.5)}, anchor = north east},
		cycle list name=exotic,
		ytick={2,3,4,5,6,7,8,9,10,11,12,13,14,15,16},
		yticklabels={,,,$5$,,,,,$10$,,,,,$15$,},
		]
		\addplot3+[thick, restrict x to domain={4:4}, unbounded coords=discard, mark=none]table[x = leafSize, y = branchFactor, z = triangleIntersectionSum, col sep=comma]{Data/sponzaSeeTable.txt};
		\addplot3+[thick, restrict x to domain={8:8}, unbounded coords=discard, mark=none]table[x = leafSize, y = branchFactor, z = triangleIntersectionSum, col sep=comma]{Data/sponzaSeeTable.txt};
		\addplot3+[thick, restrict x to domain={12:12}, unbounded coords=discard, mark=none]table[x = leafSize, y = branchFactor, z = triangleIntersectionSum, col sep=comma]{Data/sponzaSeeTable.txt};
		\addplot3+[thick, restrict x to domain={16:16}, unbounded coords=discard, mark=none]table[x = leafSize, y = branchFactor, z = triangleIntersectionSum, col sep=comma]{Data/sponzaSeeTable.txt};
		
		\legend{L4,L8,L12,L16}
		\end{axis}
		\end{tikzpicture}
	\end{minipage}\hfil \hfil
	\begin{minipage}[t]{0.4\textwidth}
		%TODO move label to the right side
		\begin{tikzpicture}
		\begin{axis}
		[
		%view={90}{0} for x, view={0}{0} for y restriction
		view={90}{0},
		xlabel = \leafs,
		ylabel = \branchf,
		zlabel = time rest,
		cycle list name=exotic,
		ytick={2,3,4,5,6,7,8,9,10,11,12,13,14,15,16},
		yticklabels={,,,$5$,,,,,$10$,,,,,$15$,},
		]
		\addplot3+[thick, restrict x to domain={4:4}, unbounded coords=discard, mark=none]table[x = leafSize, y = branchFactor, z = rayTimeWithoutTri, col sep=comma]{Data/sponzaSeeTable.txt};
		\addplot3+[thick, restrict x to domain={8:8}, unbounded coords=discard, mark=none]table[x = leafSize, y = branchFactor, z = rayTimeWithoutTri, col sep=comma]{Data/sponzaSeeTable.txt};
		\addplot3+[thick, restrict x to domain={12:12}, unbounded coords=discard, mark=none]table[x = leafSize, y = branchFactor, z = rayTimeWithoutTri, col sep=comma]{Data/sponzaSeeTable.txt};
		\addplot3+[thick, restrict x to domain={16:16}, unbounded coords=discard, mark=none]table[x = leafSize, y = branchFactor, z = rayTimeWithoutTri, col sep=comma]{Data/sponzaSeeTable.txt};
		
		\legend{L4,L8,L12,L16}
		\end{axis}
		\end{tikzpicture}
	\end{minipage}
	\caption{Comparison of L4, L8, L12, L16 for sponza performance tests. (see4)}
\end{figure}

\fi

\iffalse

\section{Second Ispc performance test}

First part is an overview of the general performance with avx. The tests go from L8 to L64 and B8 to B64. It should be noted that i only the triangle time is specific, while the node time is calculated by total raytime - triangle time

\iftrue
\begin{figure}[!htb]
	\begin{minipage}[t]{0.4\textwidth} 
		\plot{Total Time}{raytracerTotalTime}{Data/AVX/sponzaAVXSeqoptimalTableWithSpace.txt}
	\end{minipage}\hfil \hfil
	
	\caption{Overview over the sponza ispc performance tests. (avx2)}
\end{figure}

\begin{figure}[!htb]
	\begin{minipage}[t]{0.4\textwidth} 
		\plot{Time triangle intersections}{triangleIntersectionSum}{Data/AVX/sponzaAVXSeqoptimalTableWithSpace.txt}
	\end{minipage}\hfil \hfil
	\begin{minipage}[t]{0.4\textwidth}
		\plotr{Time nodes}{rayTimeSumWithoutTri}{Data/AVX/sponzaAVXSeqoptimalTableWithSpace.txt}
	\end{minipage}
	
	\caption{Overview over the sponza ispc performance tests, separated into Node and Leaf Time. (avx2)}
\end{figure}


\newpage
\fi

\pgfplotsset{
	every axis/.append style={colorbar = false},
}

\begin{figure}[!htb]
	\begin{minipage}[t]{0.4\textwidth}
		%TODO move label to the right side
		\begin{tikzpicture}
		\begin{axis}
		[
		%view={90}{0} for x, view={0}{0} for y restriction
		view={0}{0},
		xlabel = \leafs,
		ylabel = \branchf,
		zlabel = Total Time,
		cycle list name=exotic,
		legend style={at={(0.03,0.97)}, anchor = north west},
		xtick = {8, 16, ..., 128},
		%		xticklabels={$8$,$16$,$24$,$32$,$40$,$48$,$56$,$64$,$72$,$80$,$88$,$96$,$104$,$112$,$120$,$128$},
		]
		\addplot3+[thick, restrict y to domain={8: 8}, restrict x to domain = {8 : 64}, mark=none]table[x = leafSize, y = branchFactor, z = raytracerTotalTime, col sep=comma]{Data/AVX/sponzaAVXSeqoptimalTable.txt};
		\addplot3+[thick, restrict y to domain={16: 16}, restrict x to domain = {8 : 64}, mark=none]table[x = leafSize, y = branchFactor, z = raytracerTotalTime, col sep=comma]{Data/AVX/sponzaAVXSeqoptimalTable.txt};
		
		\legend{N8, N16}
		\end{axis}
		\end{tikzpicture}
	\end{minipage}\hfil \hfil
	\begin{minipage}[t]{0.4\textwidth}
		\begin{tikzpicture}
		\begin{axis}
		[
		%view={90}{0} for x, view={0}{0} for y restriction
		view={0}{0},
		xlabel = \leafs,
		ylabel = \branchf,
		zlabel = Total Time,
		cycle list name=exotic,
		legend style={at={(0.03,1.5)}, anchor = north west},
		xtick = {8, 16, ..., 64},
		%xticklabels={$8$,$16$,$24$,$32$,$40$,$48$,$56$,$64$,$72$,$80$,$88$,$96$,$104$,$112$,$120$,$128$},
		]
		\addplot3+[thick, restrict y to domain={8: 8}, mark=none]table[x = leafSize, y = branchFactor, z = raytracerTotalTime, col sep=comma]{Data/AVX/sponzaAVXSeqoptimalTable.txt};
		\addplot3+[thick, restrict y to domain={16: 16}, mark=none]table[x = leafSize, y = branchFactor, z = raytracerTotalTime, col sep=comma]{Data/AVX/sponzaAVXSeqoptimalTable.txt};
		\addplot3+[thick, restrict y to domain={24: 24}, mark=none]table[x = leafSize, y = branchFactor, z = raytracerTotalTime, col sep=comma]{Data/AVX/sponzaAVXSeqoptimalTable.txt};
		\addplot3+[thick, restrict y to domain={32: 32}, mark=none]table[x = leafSize, y = branchFactor, z = raytracerTotalTime, col sep=comma]{Data/AVX/sponzaAVXSeqoptimalTable.txt};
		\addplot3+[thick, restrict y to domain={40: 40}, mark=none]table[x = leafSize, y = branchFactor, z = raytracerTotalTime, col sep=comma]{Data/AVX/sponzaAVXSeqoptimalTable.txt};
		\addplot3+[thick, restrict y to domain={48: 48}, mark=none]table[x = leafSize, y = branchFactor, z = raytracerTotalTime, col sep=comma]{Data/AVX/sponzaAVXSeqoptimalTable.txt};
		
		\legend{N8, N16, N24, N32, N40, N48}
		\end{axis}
		\end{tikzpicture}
	\end{minipage}
	
	\caption{Comparison of N8, N16, for sponza performance tests. (avx2). }
\end{figure}

\begin{figure}[!htb]
	\begin{minipage}[t]{0.4\textwidth}
		%TODO move label to the right side
		\begin{tikzpicture}
		\begin{axis}
		[
		%view={90}{0} for x, view={0}{0} for y restriction
		view={90}{0},
		xlabel = \leafs,
		ylabel = \nodes,
		zlabel = Total Time,
		cycle list name=exotic,
		legend style={at={(0.03,0.97)}, anchor = north west},
		ytick = {8, 16, ..., 128},
		%		xticklabels={$8$,$16$,$24$,$32$,$40$,$48$,$56$,$64$,$72$,$80$,$88$,$96$,$104$,$112$,$120$,$128$},
		]
		\addplot3+[thick, restrict x to domain={8: 8},unbounded coords=discard, mark=none]table[x = leafSize, y = branchFactor, z = raytracerTotalTime, col sep=comma]{Data/AVX/sponzaAVXSeqoptimalTable.txt};
		\addplot3+[thick, restrict x to domain={16: 16},unbounded coords=discard, mark=none]table[x = leafSize, y = branchFactor, z = raytracerTotalTime, col sep=comma]{Data/AVX/sponzaAVXSeqoptimalTable.txt};
		
		\legend{N8, N16}
		\end{axis}
		\end{tikzpicture}
	\end{minipage}\hfil \hfil
	\begin{minipage}[t]{0.4\textwidth}
		\begin{tikzpicture}
		\begin{axis}
		[
		%view={90}{0} for x, view={0}{0} for y restriction
		view={90}{0},
		xlabel = \leafs,
		ylabel = \nodes,
		zlabel = Total Time,
		cycle list name=exotic,
		legend style={at={(0.03,1.5)}, anchor = north west},
		ytick = {8, 16, ..., 64},
		%xticklabels={$8$,$16$,$24$,$32$,$40$,$48$,$56$,$64$,$72$,$80$,$88$,$96$,$104$,$112$,$120$,$128$},
		]
		\addplot3+[thick, restrict x to domain={8: 8},unbounded coords=discard, mark=none]table[x = leafSize, y = branchFactor, z = raytracerTotalTime, col sep=comma]{Data/AVX/sponzaAVXSeqoptimalTable.txt};
		\addplot3+[thick, restrict x to domain={16: 16},unbounded coords=discard, mark=none]table[x = leafSize, y = branchFactor, z = raytracerTotalTime, col sep=comma]{Data/AVX/sponzaAVXSeqoptimalTable.txt};
		\addplot3+[thick, restrict x to domain={24: 24},unbounded coords=discard, mark=none]table[x = leafSize, y = branchFactor, z = raytracerTotalTime, col sep=comma]{Data/AVX/sponzaAVXSeqoptimalTable.txt};
		\addplot3+[thick, restrict x to domain={32: 32},unbounded coords=discard, mark=none]table[x = leafSize, y = branchFactor, z = raytracerTotalTime, col sep=comma]{Data/AVX/sponzaAVXSeqoptimalTable.txt};
		\addplot3+[thick, restrict x to domain={40: 40},unbounded coords=discard, mark=none]table[x = leafSize, y = branchFactor, z = raytracerTotalTime, col sep=comma]{Data/AVX/sponzaAVXSeqoptimalTable.txt};
		\addplot3+[thick, restrict x to domain={48: 48},unbounded coords=discard, mark=none]table[x = leafSize, y = branchFactor, z = raytracerTotalTime, col sep=comma]{Data/AVX/sponzaAVXSeqoptimalTable.txt};
		\addplot3+[thick, restrict x to domain={56: 56},unbounded coords=discard, mark=none]table[x = leafSize, y = branchFactor, z = raytracerTotalTime, col sep=comma]{Data/AVX/sponzaAVXSeqoptimalTable.txt};
		\addplot3+[thick, restrict x to domain={64: 64},unbounded coords=discard, mark=none]table[x = leafSize, y = branchFactor, z = raytracerTotalTime, col sep=comma]{Data/AVX/sponzaAVXSeqoptimalTable.txt};
		
		\legend{N8, N16, N24, N32, N40, N48, N56, N64}
		\end{axis}
		\end{tikzpicture}
	\end{minipage}
	
	\caption{Effect of different leaf sizes on the total ray tracer time for sponza. (avx2) }
\end{figure}

\begin{figure}[!htb]
	\begin{minipage}[t]{0.4\textwidth}
		\begin{tikzpicture}
		\begin{axis}
		[
		%view={90}{0} for x, view={0}{0} for y restriction
		view={0}{0},
		xlabel = \leafs,
		ylabel = \nodes,
		zlabel = Time triangle intersections,
		cycle list name=exotic,
		legend style={at={(0.99,1.5)}, anchor = north east},
		xtick = {8, 16, ..., 128},
		%xticklabels={$8$,$16$,$24$,$32$,$40$,$48$,$56$,$64$,$72$,$80$,$88$,$96$,$104$,$112$,$120$,$128$},
		]
		\addplot3+[thick, restrict y to domain={8: 8}, mark=none]table[x = leafSize, y = branchFactor, z = triangleIntersectionSum, col sep=comma]{Data/AVX/sponzaAVXSeqoptimalTable.txt};
		\addplot3+[thick, restrict y to domain={16: 16}, mark=none]table[x = leafSize, y = branchFactor, z = triangleIntersectionSum, col sep=comma]{Data/AVX/sponzaAVXSeqoptimalTable.txt};

		\legend{N8, N16}
		\end{axis}
		\end{tikzpicture}
	\end{minipage}\hfil \hfil
	\begin{minipage}[t]{0.4\textwidth}
		%TODO move label to the right side
		\begin{tikzpicture}
		\begin{axis}
		[
		%view={90}{0} for x, view={0}{0} for y restriction
		view={0}{0},
		xlabel = \leafs,
		ylabel = \nodes,
		zlabel = Time Nodes,
		cycle list name=exotic,
		legend style={at={(0.99,1.5)}, anchor = north east},
		xtick = {8, 16, ..., 128},
%		xticklabels={$8$,$16$,$24$,$32$,$40$,$48$,$56$,$64$,$72$,$80$,$88$,$96$,$104$,$112$,$120$,$128$},
		]
		\addplot3+[thick, restrict y to domain={8: 8}, mark=none]table[x = leafSize, y = branchFactor, z = rayTimeSumWithoutTri, col sep=comma]{Data/AVX/sponzaAVXSeqoptimalTable.txt};
		\addplot3+[thick, restrict y to domain={16: 16}, mark=none]table[x = leafSize, y = branchFactor, z = rayTimeSumWithoutTri, col sep=comma]{Data/AVX/sponzaAVXSeqoptimalTable.txt};
		\addplot3+[thick, restrict y to domain={24: 24}, mark=none]table[x = leafSize, y = branchFactor, z = rayTimeSumWithoutTri, col sep=comma]{Data/AVX/sponzaAVXSeqoptimalTable.txt};
		\addplot3+[thick, restrict y to domain={32: 32}, mark=none]table[x = leafSize, y = branchFactor, z = rayTimeSumWithoutTri, col sep=comma]{Data/AVX/sponzaAVXSeqoptimalTable.txt};
		
		\legend{N8, N16, N24, N32}
		\end{axis}
		\end{tikzpicture}
	\end{minipage}
	
	\caption{Effect of different leaf sizes on the triangle time and node time. (avx2) The leaf size influences the node time by reducing the size of the bvh. }
\end{figure}



\begin{figure}[!htb]
	\begin{minipage}[t]{0.4\textwidth}
		\begin{tikzpicture}
		\begin{axis}
		[
		%view={90}{0} for x, view={0}{0} for y restriction
		view={90}{00},
		xlabel = \leafs,
		ylabel = \nodes,
		zlabel = Time triangle intersections,
		legend style={at={(0.99,1.5)}, anchor = north east},
		cycle list name=exotic,
		ytick = {8, 16, ..., 128},
		]
		\addplot3+[thick, restrict x to domain={8:8}, unbounded coords=discard, mark=none]table[x = leafSize, y = branchFactor, z = triangleIntersectionSum, col sep=comma]{Data/AVX/sponzaAVXSeqoptimalTable.txt};
		\addplot3+[thick, restrict x to domain={16:16}, unbounded coords=discard, mark=none]table[x = leafSize, y = branchFactor, z = triangleIntersectionSum, col sep=comma]{Data/AVX/sponzaAVXSeqoptimalTable.txt};
		\addplot3+[thick, restrict x to domain={24:24}, unbounded coords=discard, mark=none]table[x = leafSize, y = branchFactor, z = triangleIntersectionSum, col sep=comma]{Data/AVX/sponzaAVXSeqoptimalTable.txt};
		\addplot3+[thick, restrict x to domain={32:32}, unbounded coords=discard, mark=none]table[x = leafSize, y = branchFactor, z = triangleIntersectionSum, col sep=comma]{Data/AVX/sponzaAVXSeqoptimalTable.txt};
		
		\legend{L8,L16,L24,L32}
		\end{axis}
		\end{tikzpicture}
	\end{minipage}\hfil \hfil
	\begin{minipage}[t]{0.4\textwidth}
		%TODO move label to the right side
		\begin{tikzpicture}
		\begin{axis}
		[
		%view={90}{0} for x, view={0}{0} for y restriction
		view={90}{0},
		xlabel = \leafs,
		ylabel = \nodes,
		zlabel = Time node,
		legend style={at={(0.99,1.5)}, anchor = north east},
		cycle list name=exotic,
		ytick = {8, 16, ..., 128},
		]
		\addplot3+[thick, restrict x to domain={8:8}, unbounded coords=discard, mark=none]table[x = leafSize, y = branchFactor, z = rayTimeSumWithoutTri, col sep=comma]{Data/AVX/sponzaAVXSeqoptimalTable.txt};
		\addplot3+[thick, restrict x to domain={8:16}, unbounded coords=discard, mark=none]table[x = leafSize, y = branchFactor, z = rayTimeSumWithoutTri, col sep=comma]{Data/AVX/sponzaAVXSeqoptimalTable.txt};
		\addplot3+[thick, restrict x to domain={24:24}, unbounded coords=discard, mark=none]table[x = leafSize, y = branchFactor, z = rayTimeSumWithoutTri, col sep=comma]{Data/AVX/sponzaAVXSeqoptimalTable.txt};
		\addplot3+[thick, restrict x to domain={32:32}, unbounded coords=discard, mark=none]table[x = leafSize, y = branchFactor, z = rayTimeSumWithoutTri, col sep=comma]{Data/AVX/sponzaAVXSeqoptimalTable.txt};
		
		\legend{L8,L16,L24,L32}
		\end{axis}
		\end{tikzpicture}
	\end{minipage}
	\caption{Effect of different Node sizes on the triangle time and node time.(avx2) Node size has basically no influence in the triangle intersections.}
\end{figure}
\fi

\iffalse

\section{First compute cost tests}

For now plots are with SSE and the San Miguel scene (16,852,353 Vertices and 5,600,315 Triangles)

\pgfplotsset{
	every axis/.append style={colorbar = false},
}
\begin{minipage}[t]{0.8\textwidth}
\begin{tikzpicture}
\pgfplotsset{ymin=0, ymax=1}
\begin{axis}[
ybar stacked,
legend style={at={(0.99,0.2)}, anchor = north east},
xlabel = \leafs,
ylabel = cost factor,
]


\addplot+[ybar]table[y = leafComputeCostNorm, x = leafSize, col sep=comma]{Data/sanMiguelSSESeqMemoryLeafComputeCostTable.txt};
\addplot+[ybar]table[y = memoryCostNorm, x = leafSize, col sep=comma]{Data/sanMiguelSSESeqMemoryLeafComputeCostTable.txt};
\legend{\strut leaf compute cost, \strut leaf memory cost}
\end{axis}
\end{tikzpicture}
\end{minipage}

\begin{minipage}[t]{0.8\textwidth}
\begin{tikzpicture}
\pgfplotsset{ymin=0, ymax=1}
\begin{axis}[
ybar stacked,
legend style={at={(0.99,0.2)}, anchor = north east},
xlabel = \nodes,
ylabel = cost factor,
]


\addplot+[ybar]table[y = nodeComputeCostNorm, x = branchFactor, col sep=comma]{Data/sanMiguelSSESeqMemoryNodeComputeCostTable.txt};
\addplot+[ybar]table[y = memoryCostNorm, x = branchFactor, col sep=comma]{Data/sanMiguelSSESeqMemoryNodeComputeCostTable.txt};
\legend{\strut node compute cost, \strut node memory cost}
\end{axis}
\end{tikzpicture}
\end{minipage}
\fi

\iffalse
\section{Scene and Camera complexity}

\pgfplotsset{
	every axis/.append style={colorbar = false},
}

This section is supposed to illustrate why we cannot really compare different scenes with each other in respect to bvh depth. Figure \ref{badExample1} and \ref{badExample2} shows the relative memory results together with the average bvh depth of the scene. For nodes it seems very random, and for the leafs the differences are very small so that might be constant?

\begin{figure}[!htb]
	\begin{minipage}[t]{0.5\textwidth}
		\begin{tikzpicture}
		\begin{axis}
		[
		%view={90}{0} for x, view={0}{0} for y restriction
		view={0}{0},
		xlabel = average Bvh depth,
		ylabel = relative memory NODE,
		cycle list name=exotic,
		legend style={at={(0.05,0.95)}, anchor = north west},
		%xtick = {0, 2, ..., 20},
		%xticklabels={$8$,$16$,$24$,$32$,$40$,$48$,$56$,$64$,$72$,$80$,$88$,$96$,$104$,$112$,$120$,$128$},
		]
		\addplot+[thick, only marks]table[x = averageBvhDepth, y = memoryRelative, col sep=comma]{Data/LaptopResultsNoSub/NodeMemorySsePerf_N4L4.txt};		

		\end{axis}
		\end{tikzpicture}
	\end{minipage}
	\label{badExample1}

	\begin{minipage}[t]{0.5\textwidth}
		\begin{tikzpicture}
		\begin{axis}
		[
		%view={90}{0} for x, view={0}{0} for y restriction
		view={0}{0},
		xlabel = average Bvh depth,
		ylabel = relative memory LEAF,
		cycle list name=exotic,
		legend style={at={(0.05,0.95)}, anchor = north west},
		%xtick = {0, 2, ..., 20},
		%xticklabels={$8$,$16$,$24$,$32$,$40$,$48$,$56$,$64$,$72$,$80$,$88$,$96$,$104$,$112$,$120$,$128$},
		]
		\addplot+[thick, only marks]table[x = averageBvhDepth, y = memoryRelative, col sep=comma]{Data/LaptopResultsNoSub/LeafMemorySsePerf_N4L4.txt};		

		\end{axis}
		\end{tikzpicture}
	\end{minipage}
	
	\label{badExample2}
\end{figure}

\newpage
\section{Subdivision test results}

The idea is to use subdivision to measure the effect of different triangle numbers and bvh depths without changing the scene.



Sponza (262,267 trinagles) with different subdivisions. From original to triangle count * 41 (so from 262,267 to 10,752,947 in steps of 262,267). The times below are normalize with 1.46683 seconds (the time for N4L4, no subdivision)
\begin{figure}[!htb]
	\begin{minipage}[t]{0.9\textwidth}
		\begin{tikzpicture}
		\begin{axis}
		[
		%view={90}{0} for x, view={0}{0} for y restriction
		view={0}{0},
		xlabel = subdivision,
		ylabel = total time normalized,
		cycle list name=linestyles*,
		legend style={at={(0.95,0.95)}, anchor = north east},
		xtick = {1, 4, ..., 41},
		%xticklabels={$8$,$16$,$24$,$32$,$40$,$48$,$56$,$64$,$72$,$80$,$88$,$96$,$104$,$112$,$120$,$128$},
		]
		\addplot+[thick, mark=none]table[discard if not={nameId}{4}, discard if not={branch}{4}, discard if not={branchMemory}{4},   x expr = \thisrow{subdivision} +1, y expr = \thisrow{rayTimeSum} / 1.46683, col sep=comma]{Data/LaptopResults/totalNodeSsePerfTable.txt};
		\addplot+[thick, mark=none]table[discard if not={nameId}{4}, discard if not={branch}{8}, discard if not={branchMemory}{8},   x expr = \thisrow{subdivision} +1, y expr = \thisrow{rayTimeSum} / 1.46683, col sep=comma]{Data/LaptopResults/totalNodeSsePerfTable.txt};
		\addplot+[thick, mark=none]table[discard if not={nameId}{4}, discard if not={branch}{12}, discard if not={branchMemory}{12}, x expr = \thisrow{subdivision} +1, y expr = \thisrow{rayTimeSum} / 1.46683, col sep=comma]{Data/LaptopResults/totalNodeSsePerfTable.txt};
		\addplot+[thick, mark=none]table[discard if not={nameId}{4}, discard if not={branch}{16}, discard if not={branchMemory}{16}, x expr = \thisrow{subdivision} +1, y expr = \thisrow{rayTimeSum} / 1.46683, col sep=comma]{Data/LaptopResults/totalNodeSsePerfTable.txt};

	
		\legend{N4L4,N8L4,N12L4,N16L4}
		\end{axis}
		\end{tikzpicture}
	\end{minipage}
	\caption{Sponza. Time values are divided by N4L4(no subdivision). A subdivision of 16 is faster than the original scene.}
\end{figure}

Its a bit unexpected to have a scene with 16 times the amount of triangles perform better than the original scene. The reason subdivision 4 and 16 have better performance than the rest might come from the way the subdivision is performed (exact explanation in the log). I also compared it to the Node intersection counts and both results have a similar trend. (no graph for now since i still have to rework the old data manager to support subdivisions).


\begin{figure}[!htb]
	\begin{minipage}[t]{0.9\textwidth}
		\begin{tikzpicture}
		\begin{axis}
		[
		%view={90}{0} for x, view={0}{0} for y restriction
		view={0}{0},
		xlabel = subdivision,
		ylabel = total time normalized,
		cycle list name=linestyles*,
		legend style={at={(0.95,0.95)}, anchor = north east},
		xtick = {1, 4, ..., 41},
		%xticklabels={$8$,$16$,$24$,$32$,$40$,$48$,$56$,$64$,$72$,$80$,$88$,$96$,$104$,$112$,$120$,$128$},
		]
		\addplot+[thick, mark=none]table[discard if not={nameId}{4}, discard if not={leaf}{4}, discard if not={leafMemory}{4},   x expr = \thisrow{subdivision} +1, y expr = \thisrow{rayTimeSum} / 1.47138, col sep=comma]{Data/LaptopResults/totalLeafSsePerfTable.txt};
		\addplot+[thick, mark=none]table[discard if not={nameId}{4}, discard if not={leaf}{8}, discard if not={leafMemory}{8},   x expr = \thisrow{subdivision} +1, y expr = \thisrow{rayTimeSum} / 1.47138, col sep=comma]{Data/LaptopResults/totalLeafSsePerfTable.txt};
		\addplot+[thick, mark=none]table[discard if not={nameId}{4}, discard if not={leaf}{12}, discard if not={leafMemory}{12}, x expr = \thisrow{subdivision} +1, y expr = \thisrow{rayTimeSum} / 1.47138, col sep=comma]{Data/LaptopResults/totalLeafSsePerfTable.txt};
		\addplot+[thick, mark=none]table[discard if not={nameId}{4}, discard if not={leaf}{16}, discard if not={leafMemory}{16}, x expr = \thisrow{subdivision} +1, y expr = \thisrow{rayTimeSum} / 1.47138, col sep=comma]{Data/LaptopResults/totalLeafSsePerfTable.txt};
		
		\legend{N4L4,N4L8,N4L12,N4L16}
		\end{axis}
		\end{tikzpicture}
	\end{minipage}
	\caption{Sponza. Time values are divided by N4L4(no subdivision). Different leaf sizes have less effect on the total time than the node sizes.}
\end{figure}

\vspace{50mm} %5mm vertical space

For Amazon lumberyard (Figure \ref{AmazonLumberyardInterior}) (1,020,903 triangles) we can observe very similar trends as in sponza in regards to increasing subdivision. A subdivision of 4 faster than without subdivision, the difference is larger than with sponza. The time we normalize with is 1.81506 seconds.

\begin{figure}[!htb]
	\begin{minipage}[t]{0.7\textwidth}
		\begin{tikzpicture}
		\begin{axis}
		[
		%view={90}{0} for x, view={0}{0} for y restriction
		view={0}{0},
		xlabel = subdivision,
		ylabel = total time normalized,
		cycle list name=linestyles*,
		legend style={at={(1.35,0.95)}, anchor = north east},
		xtick = {1, 3, ..., 21},
		%xticklabels={$8$,$16$,$24$,$32$,$40$,$48$,$56$,$64$,$72$,$80$,$88$,$96$,$104$,$112$,$120$,$128$},
		]

		\addplot+[thick, mark=none]table[discard if not={nameId}{9}, discard if not={branch}{4}, discard if not={branchMemory}{4},   x expr = \thisrow{subdivision} +1, y expr = \thisrow{rayTimeSum} / 1.79101, col sep=comma]{Data/LaptopResults/totalNodeSsePerfTable.txt};
		\addplot+[thick, mark=none]table[discard if not={nameId}{9}, discard if not={branch}{8}, discard if not={branchMemory}{8},   x expr = \thisrow{subdivision} +1, y expr = \thisrow{rayTimeSum} / 1.79101, col sep=comma]{Data/LaptopResults/totalNodeSsePerfTable.txt};
		\addplot+[thick, mark=none]table[discard if not={nameId}{9}, discard if not={branch}{12}, discard if not={branchMemory}{12}, x expr = \thisrow{subdivision} +1, y expr = \thisrow{rayTimeSum} / 1.79101, col sep=comma]{Data/LaptopResults/totalNodeSsePerfTable.txt};
		\addplot+[thick, mark=none]table[discard if not={nameId}{9}, discard if not={branch}{16}, discard if not={branchMemory}{16}, x expr = \thisrow{subdivision} +1, y expr = \thisrow{rayTimeSum} / 1.79101, col sep=comma]{Data/LaptopResults/totalNodeSsePerfTable.txt};
		
		\legend{N4L4,N8L4,N12L4,N16L4}
		\end{axis}
		\end{tikzpicture}
	\end{minipage}

	\begin{minipage}[t]{0.7\textwidth}
		\begin{tikzpicture}
		\begin{axis}
		[
		%view={90}{0} for x, view={0}{0} for y restriction
		view={0}{0},
		xlabel = subdivision,
		ylabel = total time normalized,
		cycle list name=linestyles*,
		legend style={at={(1.35,0.95)}, anchor = north east},
		xtick = {1, 3, ..., 21},
		%xticklabels={$8$,$16$,$24$,$32$,$40$,$48$,$56$,$64$,$72$,$80$,$88$,$96$,$104$,$112$,$120$,$128$},
		]
		\addplot+[thick, mark=none]table[discard if not={nameId}{9}, discard if not={leaf}{4}, discard if not={leafMemory}{4},   x expr = \thisrow{subdivision} +1, y expr = \thisrow{rayTimeSum} /  1.78588, col sep=comma]{Data/LaptopResults/totalLeafSsePerfTable.txt};
		\addplot+[thick, mark=none]table[discard if not={nameId}{9}, discard if not={leaf}{8}, discard if not={leafMemory}{8},   x expr = \thisrow{subdivision} +1, y expr = \thisrow{rayTimeSum} /  1.78588, col sep=comma]{Data/LaptopResults/totalLeafSsePerfTable.txt};
		\addplot+[thick, mark=none]table[discard if not={nameId}{9}, discard if not={leaf}{12}, discard if not={leafMemory}{12}, x expr = \thisrow{subdivision} +1, y expr = \thisrow{rayTimeSum} /  1.78588, col sep=comma]{Data/LaptopResults/totalLeafSsePerfTable.txt};
		\addplot+[thick, mark=none]table[discard if not={nameId}{9}, discard if not={leaf}{16}, discard if not={leafMemory}{16}, x expr = \thisrow{subdivision} +1, y expr = \thisrow{rayTimeSum} /  1.78588, col sep=comma]{Data/LaptopResults/totalLeafSsePerfTable.txt};
		
		\legend{N4L4,N4L8,N4L12,N4L16}
		\end{axis}
		\end{tikzpicture}
	\end{minipage}

	\caption{Amazon Lumberyard Interior. for both plots: Time values are divided by N4L4(no subdivision).}
	\label{AmazonLumberyardInterior}
\end{figure}
\newpage

\section{Relative memory for different subdivisions}

Now how does the relative memory behave with different subdivisions? (Reminder: relative memory = time to load memory / compute time)

\begin{figure}[!htb]
	\begin{minipage}[t]{0.8\textwidth}
		\begin{tikzpicture}
			\begin{axis}
			[
			%view={90}{0} for x, view={0}{0} for y restriction
			view={0}{0},
			xlabel = subdivision,
			ylabel = relative memory NODE,
			cycle list name=linestyles*,
			legend style={at={(0.95,0.95)}, anchor = north east},
			xtick = {0, 4, ..., 41},
			%xticklabels={$8$,$16$,$24$,$32$,$40$,$48$,$56$,$64$,$72$,$80$,$88$,$96$,$104$,$112$,$120$,$128$},
			]
			\addplot+[thick, mark=none]table[discard if not={nameId}{4}, x expr = \thisrow{subdivision} +1, y = memoryRelative, col sep=comma]{Data/LaptopResults/NodeMemorySsePerf_N4L4.txt};
			\addplot+[thick, mark=none]table[discard if not={nameId}{4}, x expr = \thisrow{subdivision} +1, y = memoryRelative, col sep=comma]{Data/LaptopResults/NodeMemorySsePerf_N8L4.txt};
			\addplot+[thick, mark=none]table[discard if not={nameId}{4}, x expr = \thisrow{subdivision} +1, y = memoryRelative, col sep=comma]{Data/LaptopResults/NodeMemorySsePerf_N12L4.txt};
			\addplot+[thick, mark=none]table[discard if not={nameId}{4}, x expr = \thisrow{subdivision} +1, y = memoryRelative, col sep=comma]{Data/LaptopResults/NodeMemorySsePerf_N16L4.txt};
			
			\legend{N4L4,N8L4,N12L4,N16L4}
			\end{axis}
		\end{tikzpicture}
	\end{minipage}
	\begin{minipage}[t]{0.8\textwidth}
		\begin{tikzpicture}
			\begin{axis}
			[
			%view={90}{0} for x, view={0}{0} for y restriction
			view={0}{0},
			xlabel = subdivision,
			ylabel = relative memory LEAF,
			cycle list name=linestyles*,
			legend style={at={(1.25,0.95)}, anchor = north east},
			xtick = {0, 4, ..., 41},
			%xticklabels={$8$,$16$,$24$,$32$,$40$,$48$,$56$,$64$,$72$,$80$,$88$,$96$,$104$,$112$,$120$,$128$},
			]
			\addplot+[thick, mark=none]table[discard if not={nameId}{4}, x expr = \thisrow{subdivision} +1, y = memoryRelative, col sep=comma]{Data/LaptopResults/LeafMemorySsePerf_N4L4.txt};
			\addplot+[thick, mark=none]table[discard if not={nameId}{4}, x expr = \thisrow{subdivision} +1, y = memoryRelative, col sep=comma]{Data/LaptopResults/LeafMemorySsePerf_N4L8.txt};
			\addplot+[thick, mark=none]table[discard if not={nameId}{4}, x expr = \thisrow{subdivision} +1, y = memoryRelative, col sep=comma]{Data/LaptopResults/LeafMemorySsePerf_N4L12.txt};
			\addplot+[thick, mark=none]table[discard if not={nameId}{4}, x expr = \thisrow{subdivision} +1, y = memoryRelative, col sep=comma]{Data/LaptopResults/LeafMemorySsePerf_N4L16.txt};
			
			\legend{N4L4,N4L8,N4L12,N4L16}
			\end{axis}
		\end{tikzpicture}
	\end{minipage}
	\caption{Those two graphs show how different subdivision affect relative memory time for leafs and nodes. Interesting is that for the Nodes the N12L4 line is below the N8L4 line.}
	\label{LeafNodeMemoryRelativeSponza}
\end{figure}
\newpage
\begin{figure}[!htb]
	\begin{minipage}[t]{0.8\textwidth}
		\begin{tikzpicture}
		\begin{axis}
		[
		%view={90}{0} for x, view={0}{0} for y restriction
		view={0}{0},
		xlabel = subdivision,
		ylabel = relative memory NODE,
		cycle list name=linestyles*,
		legend style={at={(1.25,0.95)}, anchor = north east},
		xtick = {0, 2, ..., 20},
		%xticklabels={$8$,$16$,$24$,$32$,$40$,$48$,$56$,$64$,$72$,$80$,$88$,$96$,$104$,$112$,$120$,$128$},
		]
		\addplot+[thick, mark=none]table[discard if not={nameId}{9}, x expr = \thisrow{subdivision} +1, y = memoryRelative, col sep=comma]{Data/LaptopResults/NodeMemorySsePerf_N4L4.txt};
		\addplot+[thick, mark=none]table[discard if not={nameId}{9}, x expr = \thisrow{subdivision} +1, y = memoryRelative, col sep=comma]{Data/LaptopResults/NodeMemorySsePerf_N8L4.txt};
		\addplot+[thick, mark=none]table[discard if not={nameId}{9}, x expr = \thisrow{subdivision} +1, y = memoryRelative, col sep=comma]{Data/LaptopResults/NodeMemorySsePerf_N12L4.txt};
		\addplot+[thick, mark=none]table[discard if not={nameId}{9}, x expr = \thisrow{subdivision} +1, y = memoryRelative, col sep=comma]{Data/LaptopResults/NodeMemorySsePerf_N16L4.txt};
		
		\legend{N4L4,N8L4,N12L4,N16L4}
		\end{axis}
		\end{tikzpicture}
	\end{minipage}
	\begin{minipage}[t]{0.8\textwidth}
		\begin{tikzpicture}
		\begin{axis}
		[
		%view={90}{0} for x, view={0}{0} for y restriction
		view={0}{0},
		xlabel = subdivision,
		ylabel = relative memory LEAF,
		cycle list name=linestyles*,
		legend style={at={(1.25,0.95)}, anchor = north east},
		xtick = {0, 2, ..., 20},
		%xticklabels={$8$,$16$,$24$,$32$,$40$,$48$,$56$,$64$,$72$,$80$,$88$,$96$,$104$,$112$,$120$,$128$},
		]
		\addplot+[thick, mark=none]table[discard if not={nameId}{9}, x expr = \thisrow{subdivision} +1, y = memoryRelative, col sep=comma]{Data/LaptopResults/LeafMemorySsePerf_N4L4.txt};
		\addplot+[thick, mark=none]table[discard if not={nameId}{9}, x expr = \thisrow{subdivision} +1, y = memoryRelative, col sep=comma]{Data/LaptopResults/LeafMemorySsePerf_N4L8.txt};
		\addplot+[thick, mark=none]table[discard if not={nameId}{9}, x expr = \thisrow{subdivision} +1, y = memoryRelative, col sep=comma]{Data/LaptopResults/LeafMemorySsePerf_N4L12.txt};
		\addplot+[thick, mark=none]table[discard if not={nameId}{9}, x expr = \thisrow{subdivision} +1, y = memoryRelative, col sep=comma]{Data/LaptopResults/LeafMemorySsePerf_N4L16.txt};
		
		\legend{N4L4,N4L8,N4L12,N4L16}
		\end{axis}
		\end{tikzpicture}
	\end{minipage}
	\caption{Same graphs as Figure \ref{LeafNodeMemoryRelativeSponza} but for amazon lumberyard interior. For the Node part the N12L4 line is also below the N8L4 line}
	\label{LeafNodeMemoryRelativeAmazon}
\end{figure}
\newpage
\begin{figure}[!htb]
	\begin{minipage}[t]{0.8\textwidth}
		\begin{tikzpicture}
		\begin{axis}
		[
		%view={90}{0} for x, view={0}{0} for y restriction
		view={0}{0},
		xlabel = average Bvh depth,
		ylabel = relative memory NODE,
		cycle list name=linestyles*,
		legend style={at={(0.05,0.95)}, anchor = north west},
		%xtick = {0, 2, ..., 20},
		%xticklabels={$8$,$16$,$24$,$32$,$40$,$48$,$56$,$64$,$72$,$80$,$88$,$96$,$104$,$112$,$120$,$128$},
		]
		\addplot+[thick, mark=none]table[discard if not={nameId}{4}, x = averageBvhDepth, y = memoryRelative, col sep=comma]{Data/LaptopResults/NodeMemorySsePerf_N4L4.txt};
		\addplot+[thick, mark=none]table[discard if not={nameId}{9}, x = averageBvhDepth, y = memoryRelative, col sep=comma]{Data/LaptopResults/NodeMemorySsePerf_N4L4.txt};

		
		\legend{sponza N4L4, amazon N4L4}
		\end{axis}
 		\end{tikzpicture}
	\end{minipage}
	\begin{minipage}[t]{0.8\textwidth}
		\begin{tikzpicture}
		\begin{axis}
		[
		%view={90}{0} for x, view={0}{0} for y restriction
		view={0}{0},
		xlabel = average Bvh depth,
		ylabel = relative memory LEAF,
		cycle list name=linestyles*,
		legend style={at={(0.05,0.95)}, anchor = north west},
		%xtick = {0, 2, ..., 20},
		%xticklabels={$8$,$16$,$24$,$32$,$40$,$48$,$56$,$64$,$72$,$80$,$88$,$96$,$104$,$112$,$120$,$128$},
		]
		\addplot+[thick, mark=none]table[discard if not={nameId}{4}, x = averageBvhDepth, y = memoryRelative, col sep=comma]{Data/LaptopResults/LeafMemorySsePerf_N4L4.txt};
		\addplot+[thick, mark=none]table[discard if not={nameId}{9}, x = averageBvhDepth, y = memoryRelative, col sep=comma]{Data/LaptopResults/LeafMemorySsePerf_N4L4.txt};

		
		\legend{sponza N4L4, amazon N4L4}
		\end{axis}
		\end{tikzpicture}
	\end{minipage}
	\caption{Now for the interesting configuration (N4L4) the relative memory time of both scenes. x axis is the avera bvh depth.}
\end{figure}



\fi

\iffalse
\section{Node Leaf intersections Lumberyard Interior}

\pgfplotsset{
	every axis/.append style={colorbar = false},
}

simple visualization of the table Lumberyard interior. 
Reminder: Node intersection is when a ray might hit a node and all the Aabbs inside this node are tested. Aabb intersections are the exact number how often we do a ray aabb test.

\begin{figure}[!htb]
	\begin{minipage}[t]{0.5\textwidth}
		\begin{tikzpicture}
		\begin{axis}
		[
		%view={90}{0} for x, view={0}{0} for y restriction
		view={0}{0},
		xlabel = Leaf size,
		ylabel = Intersections,
		cycle list name = linestyles*,
		legend style={at={(0.05,1.25)}, anchor = north west},
		%xtick = {0, 2, ..., 20},
		%xticklabels={$8$,$16$,$24$,$32$,$40$,$48$,$56$,$64$,$72$,$80$,$88$,$96$,$104$,$112$,$120$,$128$},
		]
		\addplot+[thick, mark=none]table[x = leafSize, y = primaryNodeIntersections, col sep=comma]{Data/amazonLumberyardInteriorTable.txt};		
		\addplot+[thick, mark=none]table[x = leafSize, y = primaryLeafIntersections, col sep=comma]{Data/amazonLumberyardInteriorTable.txt};
		
		\legend{Node intersections, Leaf intersections}
		\end{axis}
		\end{tikzpicture}
	\end{minipage}
	\begin{minipage}[t]{0.5\textwidth}
		\begin{tikzpicture}
		\begin{axis}
		[
		%view={90}{0} for x, view={0}{0} for y restriction
		view={0}{0},
		xlabel = Leaf size,
		%ylabel = Intersections,
		cycle list name = linestyles*,
		legend style={at={(0.05,1.25)}, anchor = north west},
		%xtick = {0, 2, ..., 20},
		%xticklabels={$8$,$16$,$24$,$32$,$40$,$48$,$56$,$64$,$72$,$80$,$88$,$96$,$104$,$112$,$120$,$128$},
		]
		%x expr = \thisrow{subdivision}
		\addplot+[thick, mark=none]table[x = leafSize, y =primaryNodeIntersections, col sep=comma]{Data/amazonLumberyardInteriorTable.txt};		
		\addplot+[thick, mark=none]table[x = leafSize, y expr = \thisrow{primaryLeafIntersections} * 4, col sep=comma]{Data/amazonLumberyardInteriorTable.txt};	
		
		\legend{Node intersections, 4 * Leaf intersections}
		\end{axis}
		\end{tikzpicture}
	\end{minipage}
	\caption{Node and Leaf Intersections}
\end{figure}

\begin{figure}[!htb]
	\begin{minipage}[t]{0.5\textwidth}
		\begin{tikzpicture}
		\begin{axis}
		[
		%view={90}{0} for x, view={0}{0} for y restriction
		view={0}{0},
		xlabel = Leaf size,
		ylabel = Intersections,
		cycle list name = linestyles*,
		legend style={at={(0.05,1.25)}, anchor = north west},
		%xtick = {0, 2, ..., 20},
		%xticklabels={$8$,$16$,$24$,$32$,$40$,$48$,$56$,$64$,$72$,$80$,$88$,$96$,$104$,$112$,$120$,$128$},
		]
		\addplot+[thick, mark=none]table[x = leafSize, y = primaryAabb, col sep=comma]{Data/amazonLumberyardInteriorTable.txt};		
		\addplot+[thick, mark=none]table[x = leafSize, y = primaryPrimitive, col sep=comma]{Data/amazonLumberyardInteriorTable.txt};	
		
		\legend{Aabb intersections, Triangle intersections}
		\end{axis}
		\end{tikzpicture}
	\end{minipage}
	\begin{minipage}[t]{0.5\textwidth}
		\begin{tikzpicture}
		\begin{axis}
		[
		%view={90}{0} for x, view={0}{0} for y restriction
		view={0}{0},
		xlabel = Leaf size,
		ylabel = Intersections,
		cycle list name = linestyles*,
		legend style={at={(0.05,1.25)}, anchor = north west},
		%xtick = {0, 2, ..., 20},
		%xticklabels={$8$,$16$,$24$,$32$,$40$,$48$,$56$,$64$,$72$,$80$,$88$,$96$,$104$,$112$,$120$,$128$},
		]
		%x expr = \thisrow{subdivision}
		\addplot+[thick, mark=none]table[x = leafSize, y =primaryAabb, col sep=comma]{Data/amazonLumberyardInteriorTable.txt};		
		\addplot+[thick, mark=none]table[x = leafSize, y expr = \thisrow{primaryPrimitive} * 4, col sep=comma]{Data/amazonLumberyardInteriorTable.txt};	
		
		\legend{Aabb intersections, 4 * Triangle intersections}
		\end{axis}
		\end{tikzpicture}
	\end{minipage}
	\caption{Aabb and Triangle intersections}
\end{figure}

\newpage

Numbers normalized by the sum of Node Intersection + Leaf Intersection of the specific Leaf size.

\begin{figure}[!htb]
	\begin{minipage}[t]{0.5\textwidth}
		\begin{tikzpicture}
		\begin{axis}
		[
		%view={90}{0} for x, view={0}{0} for y restriction
		view={0}{0},
		xlabel = Leaf size,
		ylabel = Intersections,
		cycle list name = linestyles*,
		legend style={at={(0.05,1.25)}, anchor = north west},
		%xtick = {0, 2, ..., 20},
		%xticklabels={$8$,$16$,$24$,$32$,$40$,$48$,$56$,$64$,$72$,$80$,$88$,$96$,$104$,$112$,$120$,$128$},
		]
		\addplot+[thick, mark=none]table[x = leafSize, y expr = \thisrow{primaryNodeIntersections} / ( \thisrow{primaryNodeIntersections} + \thisrow{primaryLeafIntersections}), col sep=comma]{Data/amazonLumberyardInteriorTable.txt};		
		\addplot+[thick, mark=none]table[x = leafSize, y expr = \thisrow{primaryLeafIntersections} / ( \thisrow{primaryNodeIntersections} + \thisrow{primaryLeafIntersections}), col sep=comma]{Data/amazonLumberyardInteriorTable.txt};
		
		\legend{Node intersection, Leaf intersections}
		\end{axis}
		\end{tikzpicture}
	\end{minipage}
	\begin{minipage}[t]{0.5\textwidth}
		\begin{tikzpicture}
		\begin{axis}
		[
		%view={90}{0} for x, view={0}{0} for y restriction
		view={0}{0},
		xlabel = Leaf size,
		%ylabel = Intersections,
		cycle list name = linestyles*,
		legend style={at={(0.05,1.25)}, anchor = north west},
		%xtick = {0, 2, ..., 20},
		%xticklabels={$8$,$16$,$24$,$32$,$40$,$48$,$56$,$64$,$72$,$80$,$88$,$96$,$104$,$112$,$120$,$128$},
		]
		%x expr = \thisrow{subdivision}
		\addplot+[thick, mark=none]table[x = leafSize,  y expr = \thisrow{primaryNodeIntersections} / ( \thisrow{primaryNodeIntersections} + \thisrow{primaryLeafIntersections} * 4), col sep=comma]{Data/amazonLumberyardInteriorTable.txt};		
		\addplot+[thick, mark=none]table[x = leafSize,  y expr = \thisrow{primaryLeafIntersections} * 4 / ( \thisrow{primaryNodeIntersections} + \thisrow{primaryLeafIntersections} * 4), col sep=comma]{Data/amazonLumberyardInteriorTable.txt};	
		
		\legend{Node intersections, 4 * Leaf intersections}
		\end{axis}
		\end{tikzpicture}
	\end{minipage}
	\caption{Node and Leaf Intersections normalized}
\end{figure}

\begin{figure}[!htb]
	\begin{minipage}[t]{0.5\textwidth}
		\begin{tikzpicture}
		\begin{axis}
		[
		%view={90}{0} for x, view={0}{0} for y restriction
		view={0}{0},
		xlabel = Leaf size,
		ylabel = Intersections,
		cycle list name = linestyles*,
		legend style={at={(0.05,1.25)}, anchor = north west},
		%xtick = {0, 2, ..., 20},
		%xticklabels={$8$,$16$,$24$,$32$,$40$,$48$,$56$,$64$,$72$,$80$,$88$,$96$,$104$,$112$,$120$,$128$},
		]
		\addplot+[thick, mark=none]table[x = leafSize, y expr = \thisrow{primaryAabb} / ( \thisrow{primaryAabb} + \thisrow{primaryPrimitive}), col sep=comma]{Data/amazonLumberyardInteriorTable.txt};		
		\addplot+[thick, mark=none]table[x = leafSize, y expr = \thisrow{primaryPrimitive} / ( \thisrow{primaryAabb} + \thisrow{primaryPrimitive}), col sep=comma]{Data/amazonLumberyardInteriorTable.txt};
		
		\legend{Node intersection, Leaf intersections}
		\end{axis}
		\end{tikzpicture}
	\end{minipage}
	\begin{minipage}[t]{0.5\textwidth}
		\begin{tikzpicture}
		\begin{axis}
		[
		%view={90}{0} for x, view={0}{0} for y restriction
		view={0}{0},
		xlabel = Leaf size,
		%ylabel = Intersections,
		cycle list name = linestyles*,
		legend style={at={(0.05,1.25)}, anchor = north west},
		%xtick = {0, 2, ..., 20},
		%xticklabels={$8$,$16$,$24$,$32$,$40$,$48$,$56$,$64$,$72$,$80$,$88$,$96$,$104$,$112$,$120$,$128$},
		]
		%x expr = \thisrow{subdivision}
		\addplot+[thick, mark=none]table[x = leafSize,  y expr = \thisrow{primaryAabb} / ( \thisrow{primaryAabb} + \thisrow{primaryPrimitive} * 4), col sep=comma]{Data/amazonLumberyardInteriorTable.txt};		
		\addplot+[thick, mark=none]table[x = leafSize,  y expr = \thisrow{primaryPrimitive} * 4 / ( \thisrow{primaryAabb} + \thisrow{primaryPrimitive} * 4), col sep=comma]{Data/amazonLumberyardInteriorTable.txt};	
		
		\legend{Node intersections, 4 * Leaf intersections}
		\end{axis}
		\end{tikzpicture}
	\end{minipage}
	\caption{Aabb and Triangle intersections normalized}
\end{figure}

\newpage

Now to the intersection success ratio. (how many of the triangle tests where positive). For Nodes its always about 0.3 (mostly affected by branching factor)

\begin{figure}[!htb]
	\begin{minipage}[t]{0.5\textwidth}
		\begin{tikzpicture}
		\begin{axis}
		[
		%view={90}{0} for x, view={0}{0} for y restriction
		view={0}{0},
		xlabel = Leaf size,
		ylabel = Success ratio,
		cycle list name = linestyles*,
		legend style={at={(0.05,1.25)}, anchor = north west},
		%xtick = {0, 2, ..., 20},
		%xticklabels={$8$,$16$,$24$,$32$,$40$,$48$,$56$,$64$,$72$,$80$,$88$,$96$,$104$,$112$,$120$,$128$},
		]	
		\addplot+[thick, mark=none]table[x = leafSize, y = primaryTriangleSuccessRatio, col sep=comma]{Data/amazonLumberyardInteriorTable.txt};
		
		\legend{Triangle success ratio}
		\end{axis}
		\end{tikzpicture}
	\end{minipage}
	\begin{minipage}[t]{0.5\textwidth}
		\begin{tikzpicture}
		\begin{axis}
		[
		%view={90}{0} for x, view={0}{0} for y restriction
		view={0}{0},
		xlabel = Leaf size,
		ylabel = Successfull Intersections,
		cycle list name = linestyles*,
		legend style={at={(0.05,1.25)}, anchor = north west},
		%xtick = {0, 2, ..., 20},
		%xticklabels={$8$,$16$,$24$,$32$,$40$,$48$,$56$,$64$,$72$,$80$,$88$,$96$,$104$,$112$,$120$,$128$},
		]	
		\addplot+[thick, mark=none]table[x = leafSize, y expr = \thisrow{primaryTriangleSuccessRatio} * \thisrow{primaryPrimitive}, col sep=comma]{Data/amazonLumberyardInteriorTable.txt};
		
		\legend{Number triangle hits}
		\end{axis}
		\end{tikzpicture}
	\end{minipage}
\end{figure}

So when we increase the leafsize we also get a bit more positive hits, but i think those hits are mostly within one leaf node.

\section{Node Leaf intersections Lumberyard Exterior}

Same as above but nor for the lumberyard exterior scene. The results are very similar and the lines would mostly overlap when in one graph.

\begin{figure}[!htb]
	\begin{minipage}[t]{0.5\textwidth}
		\begin{tikzpicture}
		\begin{axis}
		[
		%view={90}{0} for x, view={0}{0} for y restriction
		view={0}{0},
		xlabel = Leaf size,
		ylabel = Intersections,
		cycle list name = linestyles*,
		legend style={at={(0.05,1.25)}, anchor = north west},
		%xtick = {0, 2, ..., 20},
		%xticklabels={$8$,$16$,$24$,$32$,$40$,$48$,$56$,$64$,$72$,$80$,$88$,$96$,$104$,$112$,$120$,$128$},
		]
		\addplot+[thick, mark=none]table[x = leafSize, y = primaryNodeIntersections, col sep=comma]{Data/amazonLumberyardExteriorTable.txt};		
		\addplot+[thick, mark=none]table[x = leafSize, y = primaryLeafIntersections, col sep=comma]{Data/amazonLumberyardExteriorTable.txt};
		
		\legend{Node intersections, Leaf intersections}
		\end{axis}
		\end{tikzpicture}
	\end{minipage}
	\begin{minipage}[t]{0.5\textwidth}
		\begin{tikzpicture}
		\begin{axis}
		[
		%view={90}{0} for x, view={0}{0} for y restriction
		view={0}{0},
		xlabel = Leaf size,
		%ylabel = Intersections,
		cycle list name = linestyles*,
		legend style={at={(0.05,1.25)}, anchor = north west},
		%xtick = {0, 2, ..., 20},
		%xticklabels={$8$,$16$,$24$,$32$,$40$,$48$,$56$,$64$,$72$,$80$,$88$,$96$,$104$,$112$,$120$,$128$},
		]
		%x expr = \thisrow{subdivision}
		\addplot+[thick, mark=none]table[x = leafSize, y =primaryNodeIntersections, col sep=comma]{Data/amazonLumberyardExteriorTable.txt};		
		\addplot+[thick, mark=none]table[x = leafSize, y expr = \thisrow{primaryLeafIntersections} * 4, col sep=comma]{Data/amazonLumberyardExteriorTable.txt};	
		
		\legend{Node intersections, 4 * Leaf intersections}
		\end{axis}
		\end{tikzpicture}
	\end{minipage}
	\caption{Node and Leaf Intersections}
\end{figure}

\begin{figure}[!htb]
	\begin{minipage}[t]{0.5\textwidth}
		\begin{tikzpicture}
		\begin{axis}
		[
		%view={90}{0} for x, view={0}{0} for y restriction
		view={0}{0},
		xlabel = Leaf size,
		ylabel = Intersections,
		cycle list name = linestyles*,
		legend style={at={(0.05,1.25)}, anchor = north west},
		%xtick = {0, 2, ..., 20},
		%xticklabels={$8$,$16$,$24$,$32$,$40$,$48$,$56$,$64$,$72$,$80$,$88$,$96$,$104$,$112$,$120$,$128$},
		]
		\addplot+[thick, mark=none]table[x = leafSize, y = primaryAabb, col sep=comma]{Data/amazonLumberyardExteriorTable.txt};		
		\addplot+[thick, mark=none]table[x = leafSize, y = primaryPrimitive, col sep=comma]{Data/amazonLumberyardExteriorTable.txt};	
		
		\legend{Aabb intersections, Triangle intersections}
		\end{axis}
		\end{tikzpicture}
	\end{minipage}
	\begin{minipage}[t]{0.5\textwidth}
		\begin{tikzpicture}
		\begin{axis}
		[
		%view={90}{0} for x, view={0}{0} for y restriction
		view={0}{0},
		xlabel = Leaf size,
		ylabel = Intersections,
		cycle list name = linestyles*,
		legend style={at={(0.05,1.25)}, anchor = north west},
		%xtick = {0, 2, ..., 20},
		%xticklabels={$8$,$16$,$24$,$32$,$40$,$48$,$56$,$64$,$72$,$80$,$88$,$96$,$104$,$112$,$120$,$128$},
		]
		%x expr = \thisrow{subdivision}
		\addplot+[thick, mark=none]table[x = leafSize, y =primaryAabb, col sep=comma]{Data/amazonLumberyardExteriorTable.txt};		
		\addplot+[thick, mark=none]table[x = leafSize, y expr = \thisrow{primaryPrimitive} * 4, col sep=comma]{Data/amazonLumberyardExteriorTable.txt};	
		
		\legend{Aabb intersections, 4 * Triangle intersections}
		\end{axis}
		\end{tikzpicture}
	\end{minipage}
	\caption{Aabb and Triangle intersections}
\end{figure}

\newpage

\begin{figure}[!htb]
	\begin{minipage}[t]{0.5\textwidth}
		\begin{tikzpicture}
		\begin{axis}
		[
		%view={90}{0} for x, view={0}{0} for y restriction
		view={0}{0},
		xlabel = Leaf size,
		ylabel = Intersections,
		cycle list name = linestyles*,
		legend style={at={(0.05,1.25)}, anchor = north west},
		%xtick = {0, 2, ..., 20},
		%xticklabels={$8$,$16$,$24$,$32$,$40$,$48$,$56$,$64$,$72$,$80$,$88$,$96$,$104$,$112$,$120$,$128$},
		]
		\addplot+[thick, mark=none]table[x = leafSize, y expr = \thisrow{primaryNodeIntersections} / ( \thisrow{primaryNodeIntersections} + \thisrow{primaryLeafIntersections}), col sep=comma]{Data/amazonLumberyardExteriorTable.txt};		
		\addplot+[thick, mark=none]table[x = leafSize, y expr = \thisrow{primaryLeafIntersections} / ( \thisrow{primaryNodeIntersections} + \thisrow{primaryLeafIntersections}), col sep=comma]{Data/amazonLumberyardExteriorTable.txt};
		
		\legend{Node intersection, Leaf intersections}
		\end{axis}
		\end{tikzpicture}
	\end{minipage}
	\begin{minipage}[t]{0.5\textwidth}
		\begin{tikzpicture}
		\begin{axis}
		[
		%view={90}{0} for x, view={0}{0} for y restriction
		view={0}{0},
		xlabel = Leaf size,
		%ylabel = Intersections,
		cycle list name = linestyles*,
		legend style={at={(0.05,1.25)}, anchor = north west},
		%xtick = {0, 2, ..., 20},
		%xticklabels={$8$,$16$,$24$,$32$,$40$,$48$,$56$,$64$,$72$,$80$,$88$,$96$,$104$,$112$,$120$,$128$},
		]
		%x expr = \thisrow{subdivision}
		\addplot+[thick, mark=none]table[x = leafSize,  y expr = \thisrow{primaryNodeIntersections} / ( \thisrow{primaryNodeIntersections} + \thisrow{primaryLeafIntersections} * 4), col sep=comma]{Data/amazonLumberyardExteriorTable.txt};		
		\addplot+[thick, mark=none]table[x = leafSize,  y expr = \thisrow{primaryLeafIntersections} * 4 / ( \thisrow{primaryNodeIntersections} + \thisrow{primaryLeafIntersections} * 4), col sep=comma]{Data/amazonLumberyardExteriorTable.txt};	
		
		\legend{Node intersections, 4 * Leaf intersections}
		\end{axis}
		\end{tikzpicture}
	\end{minipage}
	\caption{Node and Leaf Intersections normalized}
\end{figure}

\begin{figure}[!htb]
	\begin{minipage}[t]{0.5\textwidth}
		\begin{tikzpicture}
		\begin{axis}
		[
		%view={90}{0} for x, view={0}{0} for y restriction
		view={0}{0},
		xlabel = Leaf size,
		ylabel = Intersections,
		cycle list name = linestyles*,
		legend style={at={(0.05,1.25)}, anchor = north west},
		%xtick = {0, 2, ..., 20},
		%xticklabels={$8$,$16$,$24$,$32$,$40$,$48$,$56$,$64$,$72$,$80$,$88$,$96$,$104$,$112$,$120$,$128$},
		]
		\addplot+[thick, mark=none]table[x = leafSize, y expr = \thisrow{primaryAabb} / ( \thisrow{primaryAabb} + \thisrow{primaryPrimitive}), col sep=comma]{Data/amazonLumberyardExteriorTable.txt};		
		\addplot+[thick, mark=none]table[x = leafSize, y expr = \thisrow{primaryPrimitive} / ( \thisrow{primaryAabb} + \thisrow{primaryPrimitive}), col sep=comma]{Data/amazonLumberyardExteriorTable.txt};
		
		\legend{Node intersection, Leaf intersections}
		\end{axis}
		\end{tikzpicture}
	\end{minipage}
	\begin{minipage}[t]{0.5\textwidth}
		\begin{tikzpicture}
		\begin{axis}
		[
		%view={90}{0} for x, view={0}{0} for y restriction
		view={0}{0},
		xlabel = Leaf size,
		%ylabel = Intersections,
		cycle list name = linestyles*,
		legend style={at={(0.05,1.25)}, anchor = north west},
		%xtick = {0, 2, ..., 20},
		%xticklabels={$8$,$16$,$24$,$32$,$40$,$48$,$56$,$64$,$72$,$80$,$88$,$96$,$104$,$112$,$120$,$128$},
		]
		%x expr = \thisrow{subdivision}
		\addplot+[thick, mark=none]table[x = leafSize,  y expr = \thisrow{primaryAabb} / ( \thisrow{primaryAabb} + \thisrow{primaryPrimitive} * 4), col sep=comma]{Data/amazonLumberyardExteriorTable.txt};		
		\addplot+[thick, mark=none]table[x = leafSize,  y expr = \thisrow{primaryPrimitive} * 4 / ( \thisrow{primaryAabb} + \thisrow{primaryPrimitive} * 4), col sep=comma]{Data/amazonLumberyardExteriorTable.txt};	
		
		\legend{Node intersections, 4 * Leaf intersections}
		\end{axis}
		\end{tikzpicture}
	\end{minipage}
	\caption{Aabb and Triangle intersections normalized}
\end{figure}


\begin{figure}[!htb]
	\begin{minipage}[t]{0.5\textwidth}
		\begin{tikzpicture}
		\begin{axis}
		[
		%view={90}{0} for x, view={0}{0} for y restriction
		view={0}{0},
		xlabel = Leaf size,
		ylabel = Success ratio,
		cycle list name = linestyles*,
		legend style={at={(0.05,1.25)}, anchor = north west},
		%xtick = {0, 2, ..., 20},
		%xticklabels={$8$,$16$,$24$,$32$,$40$,$48$,$56$,$64$,$72$,$80$,$88$,$96$,$104$,$112$,$120$,$128$},
		]	
		\addplot+[thick, mark=none]table[x = leafSize, y = primaryTriangleSuccessRatio, col sep=comma]{Data/amazonLumberyardExteriorTable.txt};
		
		\legend{Triangle success ratio}
		\end{axis}
		\end{tikzpicture}
	\end{minipage}
	\begin{minipage}[t]{0.5\textwidth}
		\begin{tikzpicture}
		\begin{axis}
		[
		%view={90}{0} for x, view={0}{0} for y restriction
		view={0}{0},
		xlabel = Leaf size,
		ylabel = Successfull Intersections,
		cycle list name = linestyles*,
		legend style={at={(0.05,1.25)}, anchor = north west},
		%xtick = {0, 2, ..., 20},
		%xticklabels={$8$,$16$,$24$,$32$,$40$,$48$,$56$,$64$,$72$,$80$,$88$,$96$,$104$,$112$,$120$,$128$},
		]	
		\addplot+[thick, mark=none]table[x = leafSize, y expr = \thisrow{primaryTriangleSuccessRatio} * \thisrow{primaryPrimitive}, col sep=comma]{Data/amazonLumberyardExteriorTable.txt};
		
		\legend{Number triangle hits}
		\end{axis}
		\end{tikzpicture}
	\end{minipage}
\end{figure}
\fi

\iffalse
\section{Node Leaf factors}

%this file needs scenes with normal , bvh , perf data to show how the sah factors behave for several N and L with different scenes (and subdivisions?)



To calculate the cost of a node in relation to the cost of a leaf we can use the following values: (Time of Node / node intersection count) / (Time of Leaf/ Leaf intersection count)

This only holds for the pc the performance data was collected. (in this case my laptop). It seems that the values are the same on most "normal" scenes.

\begin{figure}[!htb]
	\begin{minipage}[t]{0.5\textwidth}
		\begin{tikzpicture}
		\begin{axis}
		[
		xlabel = \leafs,
		ylabel = \branchf,
		colorbar style={title = Sah Node Factor},
		xtick = {4, 8, ..., 16},
		ytick = {4, 8, ..., 16},
		]
		\addplot[matrix plot*,point meta=\thisrow{sahNodeFactor}]
		table[x = leafSize, y = branchFactor, col sep=comma]{Data/NodeFactor/amazonLumberyardInteriorTableWithSpace.txt};
		\end{axis}
		\end{tikzpicture}
	\end{minipage}
\caption{Computation cost of one node relative to the computation cost of one leaf}
\end{figure}

\pgfplotsset{
	every axis/.append style={colorbar = false},
}

\begin{figure}[!htb]
	\begin{minipage}[t]{0.5\textwidth}
		\begin{tikzpicture}
		\begin{axis}
		[
		%view={90}{0} for x, view={0}{0} for y restriction
		view={0}{0},
		xlabel = Branching factor,
		ylabel = node Factor,
		cycle list name = linestyles*,
		legend style={at={(0.05,0.95)}, anchor = north west},
		xtick = {4, 8, ..., 16},
		%xticklabels={$8$,$16$,$24$,$32$,$40$,$48$,$56$,$64$,$72$,$80$,$88$,$96$,$104$,$112$,$120$,$128$},
		]
		\addplot+[thick, mark=none]table[discard if not={leafSize}{4}, x = branchFactor, y = sahNodeFactor, col sep=comma]{Data/NodeFactor/amazonLumberyardInteriorTable.txt};		
		\addplot+[thick, mark=none]table[discard if not={leafSize}{8}, x = branchFactor, y = sahNodeFactor, col sep=comma]{Data/NodeFactor/amazonLumberyardInteriorTable.txt};		
		\addplot+[thick, mark=none]table[discard if not={leafSize}{12}, x = branchFactor, y = sahNodeFactor, col sep=comma]{Data/NodeFactor/amazonLumberyardInteriorTable.txt};		
		\addplot+[thick, mark=none]table[discard if not={leafSize}{16}, x = branchFactor, y = sahNodeFactor, col sep=comma]{Data/NodeFactor/amazonLumberyardInteriorTable.txt};		
		
		\legend{L4, L8, L12, L16}
		\end{axis}
		\end{tikzpicture}
	\end{minipage}
	\begin{minipage}[t]{0.5\textwidth}
		\begin{tikzpicture}
		\begin{axis}
		[
		%view={90}{0} for x, view={0}{0} for y restriction
		view={0}{0},
		xlabel = Leaf size,
		ylabel = node Factor,
		cycle list name = linestyles*,
		legend style={at={(0.65,0.95)}, anchor = north west},
		xtick = {4, 8, ..., 16},
		%xticklabels={$8$,$16$,$24$,$32$,$40$,$48$,$56$,$64$,$72$,$80$,$88$,$96$,$104$,$112$,$120$,$128$},
		]
		%x expr = \thisrow{subdivision}
		\addplot+[thick, mark=none]table[discard if not={branchFactor}{4}, x = leafSize, y = sahNodeFactor, col sep=comma]{Data/NodeFactor/amazonLumberyardInteriorTable.txt};		
		\addplot+[thick, mark=none]table[discard if not={branchFactor}{8}, x = leafSize, y = sahNodeFactor, col sep=comma]{Data/NodeFactor/amazonLumberyardInteriorTable.txt};		
		\addplot+[thick, mark=none]table[discard if not={branchFactor}{12}, x = leafSize, y = sahNodeFactor, col sep=comma]{Data/NodeFactor/amazonLumberyardInteriorTable.txt};		
		\addplot+[thick, mark=none]table[discard if not={branchFactor}{16}, x = leafSize, y = sahNodeFactor, col sep=comma]{Data/NodeFactor/amazonLumberyardInteriorTable.txt};	
		
		\legend{N4, N8, N12, N16}
		\end{axis}
		\end{tikzpicture}
	\end{minipage}
	\caption{Same as Figure 1 but better readable}
\end{figure}

\iffalse
\begin{figure}[!htb]
	\begin{minipage}[t]{0.5\textwidth}
		\begin{tikzpicture}
		\begin{axis}
		[
		%view={90}{0} for x, view={0}{0} for y restriction
		view={0}{0},
		xlabel = Branching factor,
		ylabel = perNodeCost,
		cycle list name = linestyles*,
		legend style={at={(0.25,1.25)}, anchor = north west},
		xtick = {4, 8, ..., 16},
		%xticklabels={$8$,$16$,$24$,$32$,$40$,$48$,$56$,$64$,$72$,$80$,$88$,$96$,$104$,$112$,$120$,$128$},
		]
		
%0.019372227533326004

		\addplot+[thick, mark=none]table[discard if not={leafSize}{4}, x = branchFactor, y = perNodeCost, col sep=comma]{Data/NodeFactor/amazonLumberyardInteriorTable.txt};		
		\addplot+[thick, mark=none]table[discard if not={leafSize}{8}, x = branchFactor, y = perNodeCost, col sep=comma]{Data/NodeFactor/amazonLumberyardInteriorTable.txt};		
		\addplot+[thick, mark=none]table[discard if not={leafSize}{12}, x = branchFactor, y = perNodeCost, col sep=comma]{Data/NodeFactor/amazonLumberyardInteriorTable.txt};		
		\addplot+[thick, mark=none]table[discard if not={leafSize}{16}, x = branchFactor, y = perNodeCost, col sep=comma]{Data/NodeFactor/amazonLumberyardInteriorTable.txt};		
		
		\legend{L4, L8, L12, L16}
		\end{axis}
		\end{tikzpicture}
	\end{minipage}
	\begin{minipage}[t]{0.5\textwidth}
		\begin{tikzpicture}
		\begin{axis}
		[
		%view={90}{0} for x, view={0}{0} for y restriction
		view={0}{0},
		xlabel = Leaf size,
		ylabel = perLeafCost,
		cycle list name = linestyles*,
		legend style={at={(0.25,1.25)}, anchor = north west},
		xtick = {4, 8, ..., 16},
		%xticklabels={$8$,$16$,$24$,$32$,$40$,$48$,$56$,$64$,$72$,$80$,$88$,$96$,$104$,$112$,$120$,$128$},
		]
		
%0.033044329284923554
		
		%x expr = \thisrow{subdivision}
		\addplot+[thick, mark=none]table[discard if not={branchFactor}{4} , x = leafSize, y = perLeafCost, col sep=comma]{Data/NodeFactor/amazonLumberyardInteriorTable.txt};		
		\addplot+[thick, mark=none]table[discard if not={branchFactor}{8} , x = leafSize, y = perLeafCost, col sep=comma]{Data/NodeFactor/amazonLumberyardInteriorTable.txt};		
		\addplot+[thick, mark=none]table[discard if not={branchFactor}{12}, x = leafSize, y = perLeafCost, col sep=comma]{Data/NodeFactor/amazonLumberyardInteriorTable.txt};		
		\addplot+[thick, mark=none]table[discard if not={branchFactor}{16}, x = leafSize, y = perLeafCost, col sep=comma]{Data/NodeFactor/amazonLumberyardInteriorTable.txt};	
		
		\legend{N4, N8, N12, N16}
		\end{axis}
		\end{tikzpicture}
	\end{minipage}
	\caption{As expected the }
\end{figure}

\begin{figure}[!htb]
	\begin{minipage}[t]{0.5\textwidth}
		\begin{tikzpicture}
		\begin{axis}
		[
		%view={90}{0} for x, view={0}{0} for y restriction
		view={0}{0},
		xlabel = Branching factor,
		ylabel = node Factor,
		legend style={at={(0.05,1.25)}, anchor = north west},
		xtick = {4, 8, ..., 16},
		%xticklabels={$8$,$16$,$24$,$32$,$40$,$48$,$56$,$64$,$72$,$80$,$88$,$96$,$104$,$112$,$120$,$128$},
		]
		\addplot+[thick, mark=none]table[discard if not={leafSize}{4}, x = branchFactor, y = sahNodeFactor, col sep=comma]{Data/NodeFactor/breakfastTable.txt};		
		\addplot+[thick, mark=none]table[discard if not={leafSize}{4}, x = branchFactor, y = sahNodeFactor, col sep=comma]{Data/NodeFactor/galleryTable.txt};		
		\addplot+[thick, mark=none]table[discard if not={leafSize}{4}, x = branchFactor, y = sahNodeFactor, col sep=comma]{Data/NodeFactor/amazonLumberyardInteriorTable.txt};		
		\addplot+[thick, mark=none]table[discard if not={leafSize}{4}, x = branchFactor, y = sahNodeFactor, col sep=comma]{Data/NodeFactor/amazonLumberyardExteriorTable.txt};		
		\addplot+[thick, mark=none]table[discard if not={leafSize}{4}, x = branchFactor, y = sahNodeFactor, col sep=comma]{Data/NodeFactor/amazonLumberyardCombinedExteriorTable.txt};		
		\addplot+[thick, mark=none]table[discard if not={leafSize}{4}, x = branchFactor, y = sahNodeFactor, col sep=comma]{Data/NodeFactor/sanMiguelTable.txt};		
		
		\legend{Breakfast, Gallery}
		\end{axis}
		\end{tikzpicture}
	\end{minipage}
	\begin{minipage}[t]{0.5\textwidth}
		\begin{tikzpicture}
		\begin{axis}
		[
		%view={90}{0} for x, view={0}{0} for y restriction
		view={0}{0},
		xlabel = Leaf size,
		ylabel = node Factor,
		legend style={at={(0.05,1.25)}, anchor = north west},
		xtick = {4, 8, ..., 16},
		%xticklabels={$8$,$16$,$24$,$32$,$40$,$48$,$56$,$64$,$72$,$80$,$88$,$96$,$104$,$112$,$120$,$128$},
		]
		\addplot+[thick, mark=none]table[discard if not={branchFactor}{4}, x = leafSize, y = sahNodeFactor, col sep=comma]{Data/NodeFactor/breakfastTable.txt};		
		\addplot+[thick, mark=none]table[discard if not={branchFactor}{4}, x = leafSize, y = sahNodeFactor, col sep=comma]{Data/NodeFactor/galleryTable.txt};		
		\addplot+[thick, mark=none]table[discard if not={branchFactor}{4}, x = leafSize, y = sahNodeFactor, col sep=comma]{Data/NodeFactor/amazonLumberyardInteriorTable.txt};		
		\addplot+[thick, mark=none]table[discard if not={branchFactor}{4}, x = leafSize, y = sahNodeFactor, col sep=comma]{Data/NodeFactor/amazonLumberyardExteriorTable.txt};		
		\addplot+[thick, mark=none]table[discard if not={branchFactor}{4}, x = leafSize, y = sahNodeFactor, col sep=comma]{Data/NodeFactor/amazonLumberyardCombinedExteriorTable.txt};		
		\addplot+[thick, mark=none]table[discard if not={branchFactor}{4}, x = leafSize, y = sahNodeFactor, col sep=comma]{Data/NodeFactor/sanMiguelTable.txt};
		
		\legend{Breakfast, Gallery}
		\end{axis}
		\end{tikzpicture}
	\end{minipage}
	\caption{Aabb and Triangle intersections}
\end{figure}
\fi


\begin{figure}[!htb]
	\begin{minipage}[t]{0.5\textwidth}
		\begin{tikzpicture}
		\begin{axis}
		[
		%view={90}{0} for x, view={0}{0} for y restriction
		view={0}{0},
		xlabel = Subdivisions,
		ylabel = node Factor,
		cycle list name = linestyles*,
		legend style={at={(0.05,1.4)}, anchor = north west},
		xtick = {4, 8, ..., 16},
		%xticklabels={$8$,$16$,$24$,$32$,$40$,$48$,$56$,$64$,$72$,$80$,$88$,$96$,$104$,$112$,$120$,$128$},
		]
		\addplot+[thick, mark=none]table[discard if not={branchFactor}{4}, discard if not={leafSize}{4}, x = subdivision, y = sahNodeFactor, col sep=comma]{Data/NodeFactor/amazonLumberyardInteriorSubTable.txt};	
		\addplot+[thick, mark=none]table[discard if not={branchFactor}{8}, discard if not={leafSize}{4}, x = subdivision, y = sahNodeFactor, col sep=comma]{Data/NodeFactor/amazonLumberyardInteriorSubTable.txt};
		\addplot+[thick, mark=none]table[discard if not={branchFactor}{12}, discard if not={leafSize}{4}, x = subdivision, y = sahNodeFactor, col sep=comma]{Data/NodeFactor/amazonLumberyardInteriorSubTable.txt};
		\addplot+[thick, mark=none]table[discard if not={branchFactor}{16}, discard if not={leafSize}{4}, x = subdivision, y = sahNodeFactor, col sep=comma]{Data/NodeFactor/amazonLumberyardInteriorSubTable.txt};
	
		\legend{N4L4, N8L4, N12L4, N16L4}
		\end{axis}
		\end{tikzpicture}
	\end{minipage}
	\begin{minipage}[t]{0.5\textwidth}
		\begin{tikzpicture}
		\begin{axis}
		[
		%view={90}{0} for x, view={0}{0} for y restriction
		view={0}{0},
		xlabel = Leaf size,
		ylabel = node Factor,
		cycle list name = linestyles*,
		legend style={at={(0.05,1.4)}, anchor = north west},
		xtick = {4, 8, ..., 16},
		%xticklabels={$8$,$16$,$24$,$32$,$40$,$48$,$56$,$64$,$72$,$80$,$88$,$96$,$104$,$112$,$120$,$128$},
		]
		%x expr = \thisrow{subdivision}
		\addplot+[thick, mark=none]table[discard if not={leafSize}{4}, discard if not={branchFactor}{4}, x = subdivision, y = sahNodeFactor, col sep=comma]{Data/NodeFactor/amazonLumberyardInteriorSubTable.txt};		
		\addplot+[thick, mark=none]table[discard if not={leafSize}{8}, discard if not={branchFactor}{4}, x = subdivision, y = sahNodeFactor, col sep=comma]{Data/NodeFactor/amazonLumberyardInteriorSubTable.txt};		
		\addplot+[thick, mark=none]table[discard if not={leafSize}{12}, discard if not={branchFactor}{4}, x = subdivision, y = sahNodeFactor, col sep=comma]{Data/NodeFactor/amazonLumberyardInteriorSubTable.txt};		
		\addplot+[thick, mark=none]table[discard if not={leafSize}{16}, discard if not={branchFactor}{4}, x = subdivision, y = sahNodeFactor, col sep=comma]{Data/NodeFactor/amazonLumberyardInteriorSubTable.txt};	
		\legend{N4L4, N4L8, N4L12, N4L16}
		\end{axis}
		\end{tikzpicture}
	\end{minipage}
	\caption{Data of different subdivisions. There seems to be no significant difference to the node factor introduced by subdivision}
\end{figure}
\fi

\iffalse
\section{Work Group}

Work group of 8x8. A work group is a 8 x 8 chunk of rays and we calculate the average, max and variance for primary and secondary rays.
\input{Plots/PrimaryN4L4WorkGroupAnalysis.pgf}
\input{Plots/SecondaryN4L4WorkGroupAnalysis.pgf}

\fi

\iftrue
\section{Small Summary}

Small Summary for Lumberyard Interior intersection numbers.

\begin{figure}[!htb]
	\centering
	\begin{minipage}{.45\textwidth}
		\scalebox{0.45}{\input{Plots/AmazonLumberyardInterior_1To16PrimaryNodeIntersection.pgf}}
	\end{minipage}
	\begin{minipage}{.45\textwidth}
		\scalebox{0.45}{\input{Plots/AmazonLumberyardInterior_1To16PrimaryAabbIntersection.pgf}}
	\end{minipage}
\end{figure}

\begin{figure}[!htb]
	\centering
	\begin{minipage}{.45\textwidth}
		\scalebox{0.45}{\input{Plots/AmazonLumberyardInterior_1To16PrimaryLeafIntersection.pgf}}
	\end{minipage}
	\begin{minipage}{.45\textwidth}
		\scalebox{0.45}{\input{Plots/AmazonLumberyardInterior_1To16PrimaryTriIntersection.pgf}}
	\end{minipage}
\end{figure}

\newpage
Now to the performance results.

\begin{figure}[!htb]
	\centering
	\begin{minipage}{.45\textwidth}
		\scalebox{0.45}{\input{Plots/AmazonLumberyardInterior_4To16PerfRenderTimePerBranch.pgf}}
	\end{minipage}
	\begin{minipage}{.45\textwidth}
		\scalebox{0.45}{\input{Plots/AmazonLumberyardInterior_4To16PerfRenderTimePerLeaf.pgf}}
	\end{minipage}
\end{figure}

\begin{figure}[!htb]
	\centering
	\begin{minipage}{.45\textwidth}
		\scalebox{0.45}{\input{Plots/AmazonLumberyardInterior_4To16PerfNodeRenderTime.pgf}}
	\end{minipage}
	\begin{minipage}{.45\textwidth}
		\scalebox{0.45}{\input{Plots/AmazonLumberyardInterior_4To16PerfLeafRenderTime.pgf}}
	\end{minipage}
\end{figure}

\begin{figure}[!htb]
	\centering
	\begin{minipage}{.45\textwidth}
		\scalebox{0.45}{\input{Plots/AmazonLumberyardInterior_4To16PerfNodeFactorBranch.pgf}}
	\end{minipage}
	\begin{minipage}{.45\textwidth}
		\scalebox{0.45}{\input{Plots/AmazonLumberyardInterior_4To16PerfNodeFactorLeaf.pgf}}
	\end{minipage}
\end{figure}


\fi


%\plotAll{Data/AverageTableSorted.txt}{Average of all normalized results. Sorted}
%\newpage




%newPages so this text is below figures



\iffalse
%\plotAll{Data/shiftHappensTable.txt}{Shift happens \url{https://sketchfab.com/3d-models/shift-happens-canyon-diorama-ffd36dfbfda8432d97388988883f6295}. Low poly scene. 53,857 Vertices and 240,865 Triangles}
%\newpage
\plotAll{Data/shiftHappensTableWithSpaceSorted.txt}{Shift happens \url{https://sketchfab.com/3d-models/shift-happens-canyon-diorama-ffd36dfbfda8432d97388988883f6295}. Low poly scene. 53,857 Vertices and 240,865 Triangles. Sorted}

%\plotAll{Data/sponzaTable.txt}{Sponza \url{http://casual-effects.com/data/index.html}. Interior scene. 192,676 Vertices and 262,267 Triangles}
%\newpage
\plotAll{Data/sponzaTableWithSpaceSorted.txt}{Sponza \url{http://casual-effects.com/data/index.html}. Interior scene. 192,676 Vertices and 262,267 Triangles. Sorted}

%\plotAll{Data/rungholtTable.txt}{Rungholt \url{http://casual-effects.com/data/index.html}. Large minecraft city. 11,630,990 Vertices and 5,815,444 Triangles}
%\newpage
\plotAll{Data/rungholtTableWithSpaceSorted.txt}{Rungholt \url{http://casual-effects.com/data/index.html}. Large minecraft city. 11,630,990 Vertices and 5,815,444 Triangles. Sorted}

%\plotAll{Data/eratoTable.txt}{Erato \url{http://casual-effects.com/data/index.html}. Scan of a marble statue. 235,332 Vertices and 412,669 Triangles}
%\newpage
\plotAll{Data/eratoTableWithSpaceSorted.txt}{Erato \url{http://casual-effects.com/data/index.html}. Scan of a marble statue. 235,332 Vertices and 412,669 Triangles. Sorted}

%\plotAll{Data/AverageTable.txt}{Average of all normalized results}
%\newpage
\plotAll{Data/AverageTableWithSpaceSorted.txt}{Average of all normalized results. Sorted}
\fi

\end{document}
5